\documentclass[11pt, a4paper,parskip=half]{scrartcl}
\usepackage{standalone}
\usepackage[utf8]{inputenc}
\usepackage{geometry}
\usepackage{graphics}
\usepackage{subcaption}
\usepackage{eurosym}
\usepackage[ngerman]{babel}
\usepackage{rotating}
\usepackage{makeidx}
\usepackage[tikz]{bclogo}

\usepackage{tikz}
\usetikzlibrary{shapes.geometric,patterns,snakes}

\usepackage{hyperref}
\hypersetup{colorlinks=true, linkcolor=blue, filecolor=blue, urlcolor=blue}
\urlstyle{same}


\title{DMR -- Digital Mobile Radio}
\subtitle{Ein Mobilfunknetz für Funkamateure}
\author{Hannes Matuschek, DM3MAT,\\\texttt{dm3mat [at] darc [dot] de}}
\date{\today}

\newcommand{\repeater}[3]{%
 \node ({#1}) at ({#2}) {%
  \begin{tikzpicture}%
   \draw [black,thick] (-.25,0) -- (0,0.5) -- (0.25,0) -- (-0.25,0);%
   \draw [black,thick,domain=-45:225] plot ({0.2*cos(\x)}, {0.5+0.2*sin(\x)});%
   \draw [black,thick,domain=-45:225] plot ({0.4*cos(\x)}, {0.5+0.4*sin(\x)});%
   \node (xxx) at (0,-.2) {{#3}};%
  \end{tikzpicture}%
 } %
}

\newcommand{\activerepeater}[3]{%
 \node ({#1}) at ({#2}) {%
  \begin{tikzpicture}%
   \draw [black,thick] (-.25,0) -- (0,0.5) -- (0.25,0) -- (-0.25,0);%
   \draw [red,thick,domain=-45:225] plot ({0.2*cos(\x)}, {0.5+0.2*sin(\x)});%
   \draw [red,thick,domain=-45:225] plot ({0.4*cos(\x)}, {0.5+0.4*sin(\x)});%
   \node (xxx) at (0,-.2) {{#3}};%
  \end{tikzpicture}%
 } %
}


\newcommand{\user}[3]{%
 \node ({#1}) at ({#2}) {%
  \begin{tikzpicture}%
   \draw [black,fill=black] (-.25,0) -- (0,0.5) -- (0.25,0) -- (-0.25,0);%
   \draw [black,fill=black] (0,.5) circle (.2); %
   \node (xxx) [text width=0.6cm, align=center] at (-.35cm,-.4) {{#3}};%
  \end{tikzpicture}%
 } %
}

\newcommand{\activeuser}[3]{%
 \node ({#1}) at ({#2}) {%
  \begin{tikzpicture}%
   \draw [red,fill=red] (-.25,0) -- (0,0.5) -- (0.25,0) -- (-0.25,0);%
   \draw [red,fill=red] (0,.5) circle (.2); %
   \node (xxx) [text width=0.6cm, align=center] at (-.35cm,-.4) {{#3}};%
  \end{tikzpicture}%
 } %
}


\newenvironment{merke}{\begin{bclogo}[couleur=blue!30,arrondi=.1,logo=\bccrayon,ombre=true]{Merke}}{\end{bclogo}}   
\newenvironment{hinweis}{\begin{bclogo}[couleur=blue!30,arrondi=.1,logo=\bcinfo,ombre=true]{Hinweis}}{\end{bclogo}}   
\newenvironment{achtung}{\begin{bclogo}[couleur=red!30,arrondi=.1,logo=\bcattention,ombre=true]{Hinweis}}{\end{bclogo}}   
\newcommand{\adef}[1]{\emph{#1}\index{#1}}
\newcommand{\aref}[1]{#1\index{#1}}
\newcommand{\altdef}[2]{\emph{#1}\index{#1|seealso{#2}}}

\makeindex

\begin{document}

\begin{titlepage}
\maketitle
\vfill
\begin{abstract}
 Dieses Script soll eine Einführung in DMR (digital mobile radio) für den 
 unbedarften Funkamateur oder jeden Interessierten sein. Ich versuche dem 
 Leser Details solange zu verheimlichen, bis es absolut notwendig wird 
 diese zu erklären. Die meisten
 Einführungen in DMR, die ich bisher gesehen habe, sind eher eine lange Liste
 von Begriffserklärungen, die ohne Erfahrung mit DMR schwer zu verstehen sind. 
  
 Viel der empfundenen Komplexität von DMR, rührt aus dem Ursprung dieser 
 Technik. DMR wurde für den kommerziellen Funk auf Großveranstaltungen oder 
 großen industriellen Anlagen entwickelt, auch 
 \href{https://de.wikipedia.org/wiki/B\%C3\%BCndelfunk}{Bündelfunk} genannt. 
 Ich werde daher damit beginnen wofür DMR entwickelt wurde und fange erst dann
 an zu erklären wie DMR für den Amateurfunk eingesetzt wird.
\end{abstract}
\end{titlepage}
\pagebreak

\tableofcontents
\pagebreak

\section{Vorwissen: Relaisbetrieb} \label{sec:vorwissen} \index{Relaisbetrieb}
In diesem Abschnitt werde ich kurz den \emph{klassischen} FM-Relaisbetrieb auf VHF und UHF im Amateurfunk beschreiben. Die allermeisten lizenzierten Funkamateure werde dies noch aus der Prüfung zur Betriebstechnik oder aus eigener Erfahrung wissen. 

Wenn Sie sich aber für Amateurfunk interessieren oder selbst noch keine Erfahrung mit dem Relaisbetrieb haben, empfehle ich Ihnen diesen Abschnitt zu lesen. 

\begin{figure}[!ht]
 \centering
 \documentclass{standalone}
\newcommand{\repeater}[3]{%
 \node ({#1}) at ({#2}) {%
  \begin{tikzpicture}%
   \draw [black,thick] (-.25,0) -- (0,0.5) -- (0.25,0) -- (-0.25,0);%
   \draw [black,thick,domain=-45:225] plot ({0.2*cos(\x)}, {0.5+0.2*sin(\x)});%
   \draw [black,thick,domain=-45:225] plot ({0.4*cos(\x)}, {0.5+0.4*sin(\x)});%
   \node (xxx) at (0,-.2) {{#3}};%
  \end{tikzpicture}%
 } %
}

\newcommand{\activerepeater}[3]{%
 \node ({#1}) at ({#2}) {%
  \begin{tikzpicture}%
   \draw [black,thick] (-.25,0) -- (0,0.5) -- (0.25,0) -- (-0.25,0);%
   \draw [red,thick,domain=-45:225] plot ({0.2*cos(\x)}, {0.5+0.2*sin(\x)});%
   \draw [red,thick,domain=-45:225] plot ({0.4*cos(\x)}, {0.5+0.4*sin(\x)});%
   \node (xxx) at (0,-.2) {{#3}};%
  \end{tikzpicture}%
 } %
}


\newcommand{\user}[3]{%
 \node ({#1}) at ({#2}) {%
  \begin{tikzpicture}%
   \draw [black,fill=black] (-.25,0) -- (0,0.5) -- (0.25,0) -- (-0.25,0);%
   \draw [black,fill=black] (0,.5) circle (.2); %
   \node (xxx) [text width=0.6cm, align=center] at (-.35cm,-.4) {{#3}};%
  \end{tikzpicture}%
 } %
}

\newcommand{\activeuser}[3]{%
 \node ({#1}) at ({#2}) {%
  \begin{tikzpicture}%
   \draw [red,fill=red] (-.25,0) -- (0,0.5) -- (0.25,0) -- (-0.25,0);%
   \draw [red,fill=red] (0,.5) circle (.2); %
   \node (xxx) [text width=0.6cm, align=center] at (-.35cm,-.4) {{#3}};%
  \end{tikzpicture}%
 } %
}

\begin{document}
 \begin{tikzpicture}[every node/.style={scale=.8}]
  \activeuser{U1}{0,0}{DM3MAT};
  \user{U2}{6,0}{DL2XYZ};
  \path[->] (U1) edge node[above] {$144.500 MHz$} (U2) ;
 \end{tikzpicture}
\end{document}

 \documentclass{standalone}
\newcommand{\repeater}[3]{%
 \node ({#1}) at ({#2}) {%
  \begin{tikzpicture}%
   \draw [black,thick] (-.25,0) -- (0,0.5) -- (0.25,0) -- (-0.25,0);%
   \draw [black,thick,domain=-45:225] plot ({0.2*cos(\x)}, {0.5+0.2*sin(\x)});%
   \draw [black,thick,domain=-45:225] plot ({0.4*cos(\x)}, {0.5+0.4*sin(\x)});%
   \node (xxx) at (0,-.2) {{#3}};%
  \end{tikzpicture}%
 } %
}

\newcommand{\activerepeater}[3]{%
 \node ({#1}) at ({#2}) {%
  \begin{tikzpicture}%
   \draw [black,thick] (-.25,0) -- (0,0.5) -- (0.25,0) -- (-0.25,0);%
   \draw [red,thick,domain=-45:225] plot ({0.2*cos(\x)}, {0.5+0.2*sin(\x)});%
   \draw [red,thick,domain=-45:225] plot ({0.4*cos(\x)}, {0.5+0.4*sin(\x)});%
   \node (xxx) at (0,-.2) {{#3}};%
  \end{tikzpicture}%
 } %
}


\newcommand{\user}[3]{%
 \node ({#1}) at ({#2}) {%
  \begin{tikzpicture}%
   \draw [black,fill=black] (-.25,0) -- (0,0.5) -- (0.25,0) -- (-0.25,0);%
   \draw [black,fill=black] (0,.5) circle (.2); %
   \node (xxx) [text width=0.6cm, align=center] at (-.35cm,-.4) {{#3}};%
  \end{tikzpicture}%
 } %
}

\newcommand{\activeuser}[3]{%
 \node ({#1}) at ({#2}) {%
  \begin{tikzpicture}%
   \draw [red,fill=red] (-.25,0) -- (0,0.5) -- (0.25,0) -- (-0.25,0);%
   \draw [red,fill=red] (0,.5) circle (.2); %
   \node (xxx) [text width=0.6cm, align=center] at (-.35cm,-.4) {{#3}};%
  \end{tikzpicture}%
 } %
}

\begin{document}
 \begin{tikzpicture}[every node/.style={scale=.8}]
  \user{U1}{0,0}{DM3MAT};
  \activeuser{U2}{6,0}{DL2XYZ};
  \path[->] (U2) edge node[above] {$144.500 MHz$} (U1) ;
 \end{tikzpicture}
\end{document}

 \caption{Einfacher Simplexbetrieb, DM3MAT sendet auf der Frequenz $144.500 MHz$ direkt zu DL2XYZ. Dieser antwortet dann auf der selben Frequenz.} \label{fig:basicsimlpex}
\end{figure}

Die meisten Verbindungen zwischen zwei Funkamateuren finden im so genannten \adef{Simplexbetrieb} statt. Das heißt, die zwei Funkamateure senden und empfangen abwechselnd auf der selben Frequenz und die Verbindung zwischen ihnen ist direkt (siehe Abb. \ref{fig:basicsimlpex}). Dies funktioniert auf Kurzwelle\footnote{Als Kurzwelle oder einfach HF (\emph{high frequency}) werden Frequenzen zwischen $3MHz$ und $30MHz$ bezeichnet.} sehr gut und man kann damit weltweite Verbindungen aufbauen. 

Auf höheren Frequenzen verhält sich die Radiowelle zunehmend wie Licht und es wird auf VHF\footnote{Als VHF (\emph{very high frequency}) werden die Frequenzen zwischen $30Mhz$ und $300MHz$ bezeichnet.} und UHF\footnote{Als UHF (\emph{ultra high frequency}) werden die Frequenzen zwischen $300Mhz$ und $3000MHz$ bezeichnet.} schwierig ohne viel Aufwand\footnote{Auch auf VHF und UHF können sehr große Entfernungen überbrückt werden, nur sind dann große Richtantennen oder ein sehr hoher Standort von Nöten.} wesentlich weiter als bis zum Horizont zu gelangen. Diese Tatsache schränkt die Reichweite gerade von Handfunkgeräten stark ein. Um dennoch einen größeren Bereich überbrücken zu können, wenn man nicht gerade über einen hohen Berg mit einer großen Antenne verfügt, können sogenannte Repeater oder Relais verwendet werden. 

\adef{Repeater}\index{Relais|seealso{Repeater}} sind automatisch arbeitende Amateurfunkstationen, die meist in exponierten Lagen (hoher Berg oder hoher Turm) installiert werden, um einen möglichst großen Bereich abdecken zu können. Ihre Aufgabe ist es, Aussendungen von Funkamateuren zu empfangen und gleichzeitig wieder auszusenden. Da diese Repeater gleichzeitig empfangen und senden müssen, können sie das nicht auf der selben Frequenz tun. Daher werden diese Repeater im sogenannten \adef{Duplexbetrieb} gefahren. Das heißt, der Repeater empfängt auf einer Frequenz (der sog. \adef{Eingabefrequenz}) und sendet eben dieses empfangende Signal gleichzeitig auf einer anderen Frequenz (der sog. \adef{Ausgabefrequenz}) wieder aus. 

\begin{figure}[!ht]
 \centering
 \documentclass{standalone}
\usepackage{tikz}
\usetikzlibrary{shapes.geometric}
\newcommand{\repeater}[3]{%
 \node ({#1}) at ({#2}) {%
  \begin{tikzpicture}%
   \draw [black,thick] (-.25,0) -- (0,0.5) -- (0.25,0) -- (-0.25,0);%
   \draw [black,thick,domain=-45:225] plot ({0.2*cos(\x)}, {0.5+0.2*sin(\x)});%
   \draw [black,thick,domain=-45:225] plot ({0.4*cos(\x)}, {0.5+0.4*sin(\x)});%
   \node (xxx) at (0,-.2) {{#3}};%
  \end{tikzpicture}%
 } %
}

\newcommand{\activerepeater}[3]{%
 \node ({#1}) at ({#2}) {%
  \begin{tikzpicture}%
   \draw [black,thick] (-.25,0) -- (0,0.5) -- (0.25,0) -- (-0.25,0);%
   \draw [red,thick,domain=-45:225] plot ({0.2*cos(\x)}, {0.5+0.2*sin(\x)});%
   \draw [red,thick,domain=-45:225] plot ({0.4*cos(\x)}, {0.5+0.4*sin(\x)});%
   \node (xxx) at (0,-.2) {{#3}};%
  \end{tikzpicture}%
 } %
}


\newcommand{\user}[3]{%
 \node ({#1}) at ({#2}) {%
  \begin{tikzpicture}%
   \draw [black,fill=black] (-.25,0) -- (0,0.5) -- (0.25,0) -- (-0.25,0);%
   \draw [black,fill=black] (0,.5) circle (.2); %
   \node (xxx) [text width=0.6cm, align=center] at (-.35cm,-.4) {{#3}};%
  \end{tikzpicture}%
 } %
}

\newcommand{\activeuser}[3]{%
 \node ({#1}) at ({#2}) {%
  \begin{tikzpicture}%
   \draw [red,fill=red] (-.25,0) -- (0,0.5) -- (0.25,0) -- (-0.25,0);%
   \draw [red,fill=red] (0,.5) circle (.2); %
   \node (xxx) [text width=0.6cm, align=center] at (-.35cm,-.4) {{#3}};%
  \end{tikzpicture}%
 } %
}

\begin{document}
 \begin{tikzpicture}[every node/.style={scale=.8}]
  \activeuser{U1}{0,0}{DM3MAT};
  \activerepeater{R1}{3,1}{DL0LDS};
  \user{U2}{6,0}{DL2XYZ};
  \path[->] (U1) edge node[above,rotate=17] {$431.9625 MHz$} (R1) ;
  \path[->] (R1) edge node[above,rotate=-17] {$439.5625 MHz$} (U2);
 \end{tikzpicture}
\end{document}

 \documentclass{standalone}
\newcommand{\repeater}[3]{%
 \node ({#1}) at ({#2}) {%
  \begin{tikzpicture}%
   \draw [black,thick] (-.25,0) -- (0,0.5) -- (0.25,0) -- (-0.25,0);%
   \draw [black,thick,domain=-45:225] plot ({0.2*cos(\x)}, {0.5+0.2*sin(\x)});%
   \draw [black,thick,domain=-45:225] plot ({0.4*cos(\x)}, {0.5+0.4*sin(\x)});%
   \node (xxx) at (0,-.2) {{#3}};%
  \end{tikzpicture}%
 } %
}

\newcommand{\activerepeater}[3]{%
 \node ({#1}) at ({#2}) {%
  \begin{tikzpicture}%
   \draw [black,thick] (-.25,0) -- (0,0.5) -- (0.25,0) -- (-0.25,0);%
   \draw [red,thick,domain=-45:225] plot ({0.2*cos(\x)}, {0.5+0.2*sin(\x)});%
   \draw [red,thick,domain=-45:225] plot ({0.4*cos(\x)}, {0.5+0.4*sin(\x)});%
   \node (xxx) at (0,-.2) {{#3}};%
  \end{tikzpicture}%
 } %
}


\newcommand{\user}[3]{%
 \node ({#1}) at ({#2}) {%
  \begin{tikzpicture}%
   \draw [black,fill=black] (-.25,0) -- (0,0.5) -- (0.25,0) -- (-0.25,0);%
   \draw [black,fill=black] (0,.5) circle (.2); %
   \node (xxx) [text width=0.6cm, align=center] at (-.35cm,-.4) {{#3}};%
  \end{tikzpicture}%
 } %
}

\newcommand{\activeuser}[3]{%
 \node ({#1}) at ({#2}) {%
  \begin{tikzpicture}%
   \draw [red,fill=red] (-.25,0) -- (0,0.5) -- (0.25,0) -- (-0.25,0);%
   \draw [red,fill=red] (0,.5) circle (.2); %
   \node (xxx) [text width=0.6cm, align=center] at (-.35cm,-.4) {{#3}};%
  \end{tikzpicture}%
 } %
}

\begin{document}
 \begin{tikzpicture}[every node/.style={scale=.8}]
  \user{U1}{0,0}{DM3MAT};
  \activerepeater{R1}{3,1}{DL0LDS};
  \activeuser{U2}{6,0}{DL2XYZ};
  \path[->] (U2) edge node[above,rotate=-17] {$431.9625 MHz$} (R1) ;
  \path[->] (R1) edge node[above,rotate=17] {$439.5625 MHz$} (U1);
 \end{tikzpicture}
\end{document}

 \caption{Einfacher Repeaterbetrieb, DM3MAT sendet auf der Eingabefrequenz $431.9625 MHz$ zum Repeater (DB0LDS) und dieser setzt das empfangende Signal direct auf der Ausgabefrequenz $439.5625 MHz$ wieder ab. Auf dieser Frequenz kann DL2XYZ das umgesetzte Signal wieder empfangen.} \label{fig:basicrepeater}
\end{figure}

Für das konkrete Beispiel in Abbildung \ref{fig:basicrepeater} bedeutet das, dass DM3MAT auf der Repeatereingabefrequenz (hier $431.9625 MHz$) sendet. Dieses Signal wird vom Repeater (hier DB0LDS) empfangen und gleichzeitig wieder auf der Ausgabefrequenz (hier $439.5625 MHz$) ausgesandt. Diese Aussendung kann nun von DL2XYZ auf der Repeaterausgabefrequenz empfangen werden. Die Antwort von DL2XYZ an DM3MAT folgt den gleichen Weg, hier sendet DL2XYZ auf der Repeatereingabefrequenz und DM3MAT kann diese Aussendung auf der Repeaterausgabefrequenz empfangenen. Auf diese Wiese können zwei Funkamateure miteinander kommunizieren, auch wenn sie sich nicht direkt erreichen können. 

\subsection{Echolink} \label{sec:echolink} \index{Echolink}
Wenn zwei Funkamateure miteinander kommunizieren wollen, die sehr weit voneinander entfernt sind und somit nicht beide einen gemeinsamen Repeater erreichen können, gibt es die Möglichkeit zwei Repeater \emph{zusammenzuschalten}. 

\begin{figure}[!ht]
 \centering
 \documentclass{standalone}
\newcommand{\repeater}[3]{%
 \node ({#1}) at ({#2}) {%
  \begin{tikzpicture}%
   \draw [black,thick] (-.25,0) -- (0,0.5) -- (0.25,0) -- (-0.25,0);%
   \draw [black,thick,domain=-45:225] plot ({0.2*cos(\x)}, {0.5+0.2*sin(\x)});%
   \draw [black,thick,domain=-45:225] plot ({0.4*cos(\x)}, {0.5+0.4*sin(\x)});%
   \node (xxx) at (0,-.2) {{#3}};%
  \end{tikzpicture}%
 } %
}

\newcommand{\activerepeater}[3]{%
 \node ({#1}) at ({#2}) {%
  \begin{tikzpicture}%
   \draw [black,thick] (-.25,0) -- (0,0.5) -- (0.25,0) -- (-0.25,0);%
   \draw [red,thick,domain=-45:225] plot ({0.2*cos(\x)}, {0.5+0.2*sin(\x)});%
   \draw [red,thick,domain=-45:225] plot ({0.4*cos(\x)}, {0.5+0.4*sin(\x)});%
   \node (xxx) at (0,-.2) {{#3}};%
  \end{tikzpicture}%
 } %
}


\newcommand{\user}[3]{%
 \node ({#1}) at ({#2}) {%
  \begin{tikzpicture}%
   \draw [black,fill=black] (-.25,0) -- (0,0.5) -- (0.25,0) -- (-0.25,0);%
   \draw [black,fill=black] (0,.5) circle (.2); %
   \node (xxx) [text width=0.6cm, align=center] at (-.35cm,-.4) {{#3}};%
  \end{tikzpicture}%
 } %
}

\newcommand{\activeuser}[3]{%
 \node ({#1}) at ({#2}) {%
  \begin{tikzpicture}%
   \draw [red,fill=red] (-.25,0) -- (0,0.5) -- (0.25,0) -- (-0.25,0);%
   \draw [red,fill=red] (0,.5) circle (.2); %
   \node (xxx) [text width=0.6cm, align=center] at (-.35cm,-.4) {{#3}};%
  \end{tikzpicture}%
 } %
}

\begin{document}
 \begin{tikzpicture}[every node/.style={scale=.8}]
  \activeuser{U1}{0,0}{DM3MAT};
  \activerepeater{R1}{3,1}{DB0SP};
  \repeater{R2}{6,1}{DB0LDS};
  \user{U2}{9,0}{DL2XYZ};
  \path[->] (U1) edge node[above,rotate=17] {$DTMF: 662699$} node[below,rotate=17]{$431.825 MHz$} (R1) ;
 \end{tikzpicture}
\end{document}

 \documentclass{standalone}
\newcommand{\repeater}[3]{%
 \node ({#1}) at ({#2}) {%
  \begin{tikzpicture}%
   \draw [black,thick] (-.25,0) -- (0,0.5) -- (0.25,0) -- (-0.25,0);%
   \draw [black,thick,domain=-45:225] plot ({0.2*cos(\x)}, {0.5+0.2*sin(\x)});%
   \draw [black,thick,domain=-45:225] plot ({0.4*cos(\x)}, {0.5+0.4*sin(\x)});%
   \node (xxx) at (0,-.2) {{#3}};%
  \end{tikzpicture}%
 } %
}

\newcommand{\activerepeater}[3]{%
 \node ({#1}) at ({#2}) {%
  \begin{tikzpicture}%
   \draw [black,thick] (-.25,0) -- (0,0.5) -- (0.25,0) -- (-0.25,0);%
   \draw [red,thick,domain=-45:225] plot ({0.2*cos(\x)}, {0.5+0.2*sin(\x)});%
   \draw [red,thick,domain=-45:225] plot ({0.4*cos(\x)}, {0.5+0.4*sin(\x)});%
   \node (xxx) at (0,-.2) {{#3}};%
  \end{tikzpicture}%
 } %
}


\newcommand{\user}[3]{%
 \node ({#1}) at ({#2}) {%
  \begin{tikzpicture}%
   \draw [black,fill=black] (-.25,0) -- (0,0.5) -- (0.25,0) -- (-0.25,0);%
   \draw [black,fill=black] (0,.5) circle (.2); %
   \node (xxx) [text width=0.6cm, align=center] at (-.35cm,-.4) {{#3}};%
  \end{tikzpicture}%
 } %
}

\newcommand{\activeuser}[3]{%
 \node ({#1}) at ({#2}) {%
  \begin{tikzpicture}%
   \draw [red,fill=red] (-.25,0) -- (0,0.5) -- (0.25,0) -- (-0.25,0);%
   \draw [red,fill=red] (0,.5) circle (.2); %
   \node (xxx) [text width=0.6cm, align=center] at (-.35cm,-.4) {{#3}};%
  \end{tikzpicture}%
 } %
}

\begin{document}
 \begin{tikzpicture}[every node/.style={scale=.8}]
  \activeuser{U1}{0,0}{DM3MAT};
  \activerepeater{R1}{3,1}{DB0SP};
  \activerepeater{R2}{6,1}{DB0LDS};
  \user{U2}{9,0}{DL2XYZ};
  \path[->] (U1) edge node[above,rotate=17] {$431.825 MHz$} (R1) ;
  \path[->,dashed] (R1) edge node[above] {via Echolink} (R2) ;	
  \path[->] (R2) edge node[above,rotate=-17] {$439.150 MHz$} (U2) ;
 \end{tikzpicture}
\end{document}

 \documentclass{standalone}
\usepackage{tikz}
\usetikzlibrary{shapes.geometric}
\newcommand{\repeater}[3]{%
 \node ({#1}) at ({#2}) {%
  \begin{tikzpicture}%
   \draw [black,thick] (-.25,0) -- (0,0.5) -- (0.25,0) -- (-0.25,0);%
   \draw [black,thick,domain=-45:225] plot ({0.2*cos(\x)}, {0.5+0.2*sin(\x)});%
   \draw [black,thick,domain=-45:225] plot ({0.4*cos(\x)}, {0.5+0.4*sin(\x)});%
   \node (xxx) at (0,-.2) {{#3}};%
  \end{tikzpicture}%
 } %
}

\newcommand{\activerepeater}[3]{%
 \node ({#1}) at ({#2}) {%
  \begin{tikzpicture}%
   \draw [black,thick] (-.25,0) -- (0,0.5) -- (0.25,0) -- (-0.25,0);%
   \draw [red,thick,domain=-45:225] plot ({0.2*cos(\x)}, {0.5+0.2*sin(\x)});%
   \draw [red,thick,domain=-45:225] plot ({0.4*cos(\x)}, {0.5+0.4*sin(\x)});%
   \node (xxx) at (0,-.2) {{#3}};%
  \end{tikzpicture}%
 } %
}


\newcommand{\user}[3]{%
 \node ({#1}) at ({#2}) {%
  \begin{tikzpicture}%
   \draw [black,fill=black] (-.25,0) -- (0,0.5) -- (0.25,0) -- (-0.25,0);%
   \draw [black,fill=black] (0,.5) circle (.2); %
   \node (xxx) [text width=0.6cm, align=center] at (-.35cm,-.4) {{#3}};%
  \end{tikzpicture}%
 } %
}

\newcommand{\activeuser}[3]{%
 \node ({#1}) at ({#2}) {%
  \begin{tikzpicture}%
   \draw [red,fill=red] (-.25,0) -- (0,0.5) -- (0.25,0) -- (-0.25,0);%
   \draw [red,fill=red] (0,.5) circle (.2); %
   \node (xxx) [text width=0.6cm, align=center] at (-.35cm,-.4) {{#3}};%
  \end{tikzpicture}%
 } %
}

\begin{document}
 \begin{tikzpicture}[every node/.style={scale=.8}]
  \user{U1}{0,0}{DM3MAT};
  \activerepeater{R1}{3,1}{DB0SP};
  \activerepeater{R2}{6,1}{DB0LDS};
  \activeuser{U2}{9,0}{DL2XYZ};
  \path[->] (R1) edge node[above,rotate=17] {$439.425 MHz$} (U1) ;
  \path[->,dashed] (R2) edge node[above] {via Echolink} (R1) ;	
  \path[->] (U2) edge node[above,rotate=-17] {$431.875 MHz$} (R2) ;
 \end{tikzpicture}
\end{document}

 \caption{Repeaterbetrieb mit Echolink. DM3MAT verbindet die Repeater DB0SP (bei Berlin) und DB0LEI (bei Leipzig) per Echolink. Daraufhin können DM3MAT und DL2XYZ wie über einen gemeinsamen Repeater kommunizieren.} \label{fig:echolink}
\end{figure}

Diese Möglichkeit nennt sich \href{http://www.echolink.org/}{Echolink}. Dieses Netzwerk erlaubt es FM Repeater per Internet miteinander zu verbinden oder sich per Internet als einzelner Teilnehmer direkt mit einem Repeater zu verbinden. Viele FM Repeater sind in diesem Netzwerk zusammengeschlossen. 

Es ist auch häufig möglich\footnote{Dies hängt von der Konfiguration des Repeaters ab.} per Funk einen Repeater zu steuern und ihn mit einem anderen Repeater via Echolink zu verbinden. Dazu wird die sogenannte Echolink Nummer des Ziel Repeaters per DTMF Tonwahl an den Quellrepeater gesandt. Dies ist in Abbildung \ref{fig:echolink} (Oben) dargestellt. Hier sendet DM3MAT die Echolink Nummer 662699 des Relais DB0LEI bei Leipzig per DTMF an den Repeater DB0SP nahe Berlin. Dieser (DB0SP) verbindet sich dann mit dem Zielrepeater bei Leipzig (DB0LEI) über das Echolink Netzwerk. Alle weiteren Aussendungen die der Quellrepeater (DB0SP) nun empfängt werden nicht nur lokal auf der Ausgabefrequenz ausgesandt, sonder werden auch am Zielrepeater bei Leipzig (DB0LEI) ausgesandt (Abb. \ref{fig:echolink} Mitte). Somit kann DL2XYZ in Leipzig DM3MAT hören. Ebenso werden alle Aussendungen die der Zielrepeater (DB0LEI) empfängt via Echolink zum Quellrepeater bei Berlin übertragen und auch dort ausgesandt (Abb. \ref{fig:echolink} Unten).  Auf diese weise können zwei Funkamateure (in diesem Beispiel DM3MAT \& DL2XYZ), die sich nicht in der Nähe des selben Repeaters befinden, dennoch miteinander kommunizieren. 

\begin{merke}
Sobald zwei Repeater per Echolink miteinander verbunden sind, verhalten sich beide wie ein einziger Repeater. 
\end{merke}

Es gibt überall auf der Welt FM Repeater die per Echolink erreichbar sind. Dadurch ist es möglich jederzeit weltweite Kontakte mit einfachsten Mitteln (FM Handfunkgeräte mit DTMF Funktion sind ab ca. \EUR{40} erhältlich) herzustellen.

\section{DMR Einführung \& Ursprung} \label{sec:ursprung}
DMR kurz für \emph{digital mobile radio} ist ein digitaler Funkstandard für Sprech- und Datenfunk. Das heißt, die Sprache wird nicht direkt über FM o.a. auf einem Kanal übertragen, sondern erst digitalisiert und mit einem Verlustbehafteten Codec codiert und dann als Datenpaket übertragen. Dies ermöglicht es bei jedem Ruf zusätzliche Informationen wie Quelle und Ziel des Rufs mitzuübertragen.

DMR wurde als Ersatz für den analogen Bündelfunk in der kommerziellen Anwendung entwickelt. Ein klassisches Beispiel für den kommerziellen Einsatz von DMR wäre ein Verkehrsflughafen. Damit ist nicht der Flugfunk auf dem Feld und in der Luft gemeint, sondern der Funkbetrieb zwischen dem ganzen Bodenpersonal. 

Auf so einem Flughafen arbeiten sehr viele Leute mit sehr unterschiedlichen Aufgaben. Da hätten wir (ohne Anspruch auf Vollständigkeit)
\begin{itemize}
 \item Die Reinigungskolonne,
 \item die Sicherheitsleute wie Gepäckkontrolle oder Wachschutz,
 \item das Vorfeld, also die Betankung, die Gepäckverladung \& das Catering, 
 \item die Betriebsfeuerwehr und
 \item die Zentrale.
\end{itemize}

All diese Mitarbeiter bekommen ein Funkgerät und sollen die folgenden Möglichkeiten haben:
\begin{itemize}
 \item Direkte Kommunikation zur Zentrale, alle Personen sollen die Zentrale erreichen können.
 \item Direkte Kommunikation zwischen zwei Personen innerhalb ihrer Gruppe ohne das andere Gruppen gestört werden. D.h., die Reinigungskolonne sollte sich untereinander absprechen können ohne die Betriebsfeuerwehr zu stören.
 \item Sog. Gruppenrufe einer Person an eine ganze Gruppe. Zum Beispiel die Zentrale an die gesamte Betriebsfeuerwehr. Aber auch ein Anruf eines Wachschützers an alle anderen Wachschützer um zum Beispiel Hilfe anzufordern. 
 \item Die Alarmierung Aller, einer ganzen Gruppe oder einzelner Personen durch die Zentrale (wird im Amateurfunk nicht verwendet).
\end{itemize}
 
Gleichzeitig ist so ein Flughafen ein riesiges Gelände. D.h., nicht alle Mitarbeiter können alle anderen Mitarbeiter direkt erreichen. Es müssen also Repeater aufgestellt werden, damit das gesamte Gelände per Funk abgedeckt ist, vor allem die Innenräume. Daher wird häufig in jedem Gebäude mindestens ein Repeater aufgestellt. 

Vergleicht man nun die Ansprüche dies Kommunikationsnetzes mit dem klassischen FM-Repeaterbetrieb (Abs. \ref{sec:vorwissen}) wird schnell deutlich, dass es sehr schwierig wird dieses Konzept per analog FM-Repeater umzusetzen, vor allem wenn mehrere Repeater in einem Netz (ähnlich Echolink) verbunden sind. Jede Kommunikation zwischen zwei Personen würde dann aber das gesamte Kommunikationsnetz belegen. 

Besser wäre es, wenn nur jene Repeater aktiv würden, die für die Kommunikation zwischen zwei Teilnehmern nötig sind. Dann stünden alle anderen Repeater für weitere Verbindungen bereit. Dieses Routing von Verbindungen sollte aber automatisch geschehen, da die zwei Teilnehmer nicht immer wissen werden, wo sich die jeweils andere Person befindet und somit mit welchem Repeater sie sich verbinden müssen. 

Um solche komplexen Kommunikationsnetze realisieren zu können ohne, dass der Teilnehmer detailliertes Wissen über dessen physische Struktur\footnote{Wissen darüber wo sich welcher Repeater befindet und wo sich welche Teilnehemer aufhalten.} benötigen, wurde DMR entwickelt.

\begin{merke}
 DMR hat mehr Ähnlichkeit mit einem Telefonnetz mit zusätzlichen Gruppenruf (anstatt ausschließlich eins-zu-eins Verbindungen) als mit klassischem FM-Repeaterbetrieb.
\end{merke} 

Das heißt, jeder Teilnehmer und damit dessen Funkgerät besitzt eine eindeutige Nummer\index{DMR-ID}. Diese Nummer liegt im Bereich $1$--$16777215$. Und wie bei einem gewöhnlichen Telefonnetz, kann ein Teilnehmer einen Anderen mit seiner Nummer direkt anrufen. Dies wird \adef{Direktruf} oder auch \adef{Private Call} genannt.

Außerdem werden Gruppen definiert, die auch wieder ihre eigene Nummer erhalten. Die sog. \adef{Sprechgruppe} oder auch \adef{Talk Group} (\emph{TG}). Diese Sprechgruppen dienen dazu alle Mitarbeiter einer bestimmten Gruppe (z.B., den Wachschutz, die Betriebsfeuerwehr, etc.) gleichzeitig erreichen zu können. Das heißt, das Funkgerät einer Reinigungskraft muss wissen, dass es auf die Gruppenrufe der Sprachgruppe \emph{Reinigung} reagieren muss aber alle anderen Sprechgruppen ignorieren soll. 

\begin{merke}
 Dieser Punkt ist sehr wichtig: Das DMR Netz selbst weiß nicht, welcher Teilnehmer zu welcher Gruppe gehört oder nicht. Das Funkgerät des Teilnehmers wird so konfiguriert, dass es nur auf bestimmte Gruppenrufe reagiert.
\end{merke}

\begin{figure}[!ht]
 \centering
 \documentclass{standalone}
\usepackage{tikz}
\usetikzlibrary{shapes.geometric}
\newcommand{\repeater}[3]{%
 \node ({#1}) at ({#2}) {%
  \begin{tikzpicture}%
   \draw [black,thick] (-.25,0) -- (0,0.5) -- (0.25,0) -- (-0.25,0);%
   \draw [black,thick,domain=-45:225] plot ({0.2*cos(\x)}, {0.5+0.2*sin(\x)});%
   \draw [black,thick,domain=-45:225] plot ({0.4*cos(\x)}, {0.5+0.4*sin(\x)});%
   \node (xxx) at (0,-.2) {{#3}};%
  \end{tikzpicture}%
 } %
}

\newcommand{\activerepeater}[3]{%
 \node ({#1}) at ({#2}) {%
  \begin{tikzpicture}%
   \draw [black,thick] (-.25,0) -- (0,0.5) -- (0.25,0) -- (-0.25,0);%
   \draw [red,thick,domain=-45:225] plot ({0.2*cos(\x)}, {0.5+0.2*sin(\x)});%
   \draw [red,thick,domain=-45:225] plot ({0.4*cos(\x)}, {0.5+0.4*sin(\x)});%
   \node (xxx) at (0,-.2) {{#3}};%
  \end{tikzpicture}%
 } %
}


\newcommand{\user}[3]{%
 \node ({#1}) at ({#2}) {%
  \begin{tikzpicture}%
   \draw [black,fill=black] (-.25,0) -- (0,0.5) -- (0.25,0) -- (-0.25,0);%
   \draw [black,fill=black] (0,.5) circle (.2); %
   \node (xxx) [text width=0.6cm, align=center] at (-.35cm,-.4) {{#3}};%
  \end{tikzpicture}%
 } %
}

\newcommand{\activeuser}[3]{%
 \node ({#1}) at ({#2}) {%
  \begin{tikzpicture}%
   \draw [red,fill=red] (-.25,0) -- (0,0.5) -- (0.25,0) -- (-0.25,0);%
   \draw [red,fill=red] (0,.5) circle (.2); %
   \node (xxx) [text width=0.6cm, align=center] at (-.35cm,-.4) {{#3}};%
  \end{tikzpicture}%
 } %
}

\begin{document}
 \begin{tikzpicture}[every node/.style={scale=.8}]
  \user{r1}{ 0,0}{Clean 1};
  \user{r2}{ 2,0}{Clean 2};	
  \draw[dotted] (3,4) -- (3,-1);
  \user{s1}{ 4,0}{Security 1};
  \user{z} { 6,0}{HQ};
  \draw[dotted] (7,4) -- (7,-1);
  \user{s2}{ 8,0}{Security 2};
  \user{r3}{10,0}{Clean 3};
  \repeater{R1}{1,3}{Terminal 1, TG: C,S};
  \repeater{R2}{5,3}{Terminal 2, TG: C,S};
  \repeater{R3}{9,3}{Apron, TG: S};
 \end{tikzpicture}
\end{document}

 \caption{Ein Beispielnetzwerk für den hypothetischen Flughafen. Es gibt drei Reinigungskräfte, zwei Sicherheitsleute und eine Zentrale. Um das gesammte Gelände abdecken zu können, werden drei Repeater benötigt einer in Terminal 1, einer in Terminal 2 und einer im Vorfeld.} \label{fig:exnet1}
\end{figure}

In Abbildung \ref{fig:exnet1} sei ein Beispielnetzwerk für den Flughafen dargestellt (in Wirklichkeit viel größer und komplexer). Nun stellen wir uns die Situation vor, dass die Reinigungskraft 1 \& 3 miteinander Sprechen wollen und gleichzeitig die Zentrale mit \emph{Sicherheit 1}. In einem einfachen analog Netz, bei dem alle Repeater einfach zusammengeschaltet wären, würde das Gespräch zwischen \emph{Reinigung 1} \& \emph{3} das gesamte Netz blockieren und die Verbindung zwischen \emph{Zentrale} und \emph{Sicherheit 1} wäre nicht möglich. 

\begin{figure}[!ht]
 \centering
 \documentclass{standalone}
\newcommand{\repeater}[3]{%
 \node ({#1}) at ({#2}) {%
  \begin{tikzpicture}%
   \draw [black,thick] (-.25,0) -- (0,0.5) -- (0.25,0) -- (-0.25,0);%
   \draw [black,thick,domain=-45:225] plot ({0.2*cos(\x)}, {0.5+0.2*sin(\x)});%
   \draw [black,thick,domain=-45:225] plot ({0.4*cos(\x)}, {0.5+0.4*sin(\x)});%
   \node (xxx) at (0,-.2) {{#3}};%
  \end{tikzpicture}%
 } %
}

\newcommand{\activerepeater}[3]{%
 \node ({#1}) at ({#2}) {%
  \begin{tikzpicture}%
   \draw [black,thick] (-.25,0) -- (0,0.5) -- (0.25,0) -- (-0.25,0);%
   \draw [red,thick,domain=-45:225] plot ({0.2*cos(\x)}, {0.5+0.2*sin(\x)});%
   \draw [red,thick,domain=-45:225] plot ({0.4*cos(\x)}, {0.5+0.4*sin(\x)});%
   \node (xxx) at (0,-.2) {{#3}};%
  \end{tikzpicture}%
 } %
}


\newcommand{\user}[3]{%
 \node ({#1}) at ({#2}) {%
  \begin{tikzpicture}%
   \draw [black,fill=black] (-.25,0) -- (0,0.5) -- (0.25,0) -- (-0.25,0);%
   \draw [black,fill=black] (0,.5) circle (.2); %
   \node (xxx) [text width=0.6cm, align=center] at (-.35cm,-.4) {{#3}};%
  \end{tikzpicture}%
 } %
}

\newcommand{\activeuser}[3]{%
 \node ({#1}) at ({#2}) {%
  \begin{tikzpicture}%
   \draw [red,fill=red] (-.25,0) -- (0,0.5) -- (0.25,0) -- (-0.25,0);%
   \draw [red,fill=red] (0,.5) circle (.2); %
   \node (xxx) [text width=0.6cm, align=center] at (-.35cm,-.4) {{#3}};%
  \end{tikzpicture}%
 } %
}

\begin{document}
 \begin{tikzpicture}[every node/.style={scale=.8}]
  \activeuser{r1}{ 0,0}{Reinigung 1};
  \user{r2}{ 2,0}{Reinigung 2};	
  \draw[dotted] (3,4) -- (3,-1);
  \activeuser{s1}{ 4,0}{Sicherheit 1};
  \activeuser{z} { 6,0}{Zentrale};
  \draw[dotted] (7,4) -- (7,-1);
  \user{s2}{ 8,0}{Sicherheit 2};
  \activeuser{r3}{10,0}{Reinigung 3};
  \activerepeater{R1}{1,3.5}{Terminal 1, TG: R,S};
  \activerepeater{R2}{5,3.5}{Terminal 2, TG: R,S};
  \activerepeater{R3}{9,3.5}{Vorfeld, TG: S};
  \draw[->] (r1) -- node[above,rotate=74] {PC: Reinigung 3} (R1);
  \path[->,dashed] (R1) edge [bend left] node[above] {via Netzwerk} (R3);
  \draw[->] (R3) -- node[above,rotate=-74] {PC: Reinigung 3} (r3);
  \draw[->] (z) -- node[above,rotate=-74] {PC: Sicherheit 1} (R2);
  \draw[->] (R2) -- node[above,rotate=74] {PC: Sicherheit 1} (s1);
 \end{tikzpicture}
\end{document}

 \caption{Zwei gleichzeitige Direktrufe (Private Calls, PC) in dem Beispielnetzwerk zwischen \emph{Reinigung 1 \& 3} sowie zwischen \emph{Zentrale} und \emph{Sicherheit 1}} \label{fig:exnet2}
\end{figure}

In einem DMR Netz hingegen, werden für einen Direktruf (Privat Call) nur jene Repeater verwendet, die dafür nötig sind. Dies ist in Abbildung \ref{fig:exnet2} zu sehen: \emph{Reinigung 1} startet einen Direktruf (Private Call) über ihren lokalen Repeater in \emph{Terminal 1}. Da das DMR Netzwerk weiß über welchen Repeater \emph{Reinigung 3} zuletzt aktiv war, wird der Direktruf vom DMR Netz über den Repeater auf dem Vorfeld etabliert. Der Repeater im Terminal 2 hingegen wird für diesen Direktruf nicht aktiv. Daher steht dieser Repeater weiterhin für weitere Kommunikation zur Verfügung. Dies nutzt die Zentrale um \emph{Sicherheit 1} per Direktruf zu erreichen. 

Solange das Gespräch zwischen \emph{Reinigung 1 \& 3} anhält sind aber die Repeater im Terminal 1 und auf dem Vorfeld belegt. D.h., die Zentrale kann \emph{Reinigung 2} und \emph{Sicherheit 2} nicht erreichen. 

Dies klingt schlimmer als es ist. Im Gegensatz zu klassischen Telefonaten gilt im DMR Netz ein Direktruf als unterbrochen sobald ein Teilnehmer die PTT Taste loslässt. Daher kann die Zentrale in den Umschaltpausen des Gespräches \emph{dazwischenrufen} und z.B. \emph{Sicherheit 2} erreichen. 

Im nächsten Beispiel (Abbildung \ref{fig:exnet3}) will die Zentrale alle Reinigungskräfte erreichen. Dazu macht sie einen Gruppenruf zur Sprechgruppe/Talk Group \emph{Reinigung} (R für Reinigung, S für Sicherheit). Damit erreicht sie die \emph{Reinigung 1 \& 2} problemlos, aber \emph{Reinigung 3} empfängt diesen Gruppenruf nicht. 

Dies liegt daran, dass das DMR Netz nicht weiß, welche Personen zu welcher Gruppe gehören. Da sich Reinigungskräfte üblicherweise nicht auf dem Vorfeld herumtreiben, hat der Repeater auf dem Vorfeld die Sprechgruppe \emph{Reinigung (R)} nicht \emph{abonniert} und leitet daher keine Gruppenrufe für diese Sprechgruppe weiter. 

\begin{figure}[p]
 \begin{subfigure}{\linewidth}
  \centering
  \documentclass{standalone}
\usepackage{tikz}
\usetikzlibrary{shapes.geometric}
\newcommand{\repeater}[3]{%
 \node ({#1}) at ({#2}) {%
  \begin{tikzpicture}%
   \draw [black,thick] (-.25,0) -- (0,0.5) -- (0.25,0) -- (-0.25,0);%
   \draw [black,thick,domain=-45:225] plot ({0.2*cos(\x)}, {0.5+0.2*sin(\x)});%
   \draw [black,thick,domain=-45:225] plot ({0.4*cos(\x)}, {0.5+0.4*sin(\x)});%
   \node (xxx) at (0,-.2) {{#3}};%
  \end{tikzpicture}%
 } %
}

\newcommand{\activerepeater}[3]{%
 \node ({#1}) at ({#2}) {%
  \begin{tikzpicture}%
   \draw [black,thick] (-.25,0) -- (0,0.5) -- (0.25,0) -- (-0.25,0);%
   \draw [red,thick,domain=-45:225] plot ({0.2*cos(\x)}, {0.5+0.2*sin(\x)});%
   \draw [red,thick,domain=-45:225] plot ({0.4*cos(\x)}, {0.5+0.4*sin(\x)});%
   \node (xxx) at (0,-.2) {{#3}};%
  \end{tikzpicture}%
 } %
}


\newcommand{\user}[3]{%
 \node ({#1}) at ({#2}) {%
  \begin{tikzpicture}%
   \draw [black,fill=black] (-.25,0) -- (0,0.5) -- (0.25,0) -- (-0.25,0);%
   \draw [black,fill=black] (0,.5) circle (.2); %
   \node (xxx) [text width=0.6cm, align=center] at (-.35cm,-.4) {{#3}};%
  \end{tikzpicture}%
 } %
}

\newcommand{\activeuser}[3]{%
 \node ({#1}) at ({#2}) {%
  \begin{tikzpicture}%
   \draw [red,fill=red] (-.25,0) -- (0,0.5) -- (0.25,0) -- (-0.25,0);%
   \draw [red,fill=red] (0,.5) circle (.2); %
   \node (xxx) [text width=0.6cm, align=center] at (-.35cm,-.4) {{#3}};%
  \end{tikzpicture}%
 } %
}

\begin{document}
  \begin{tikzpicture}[every node/.style={scale=.8}]
   \activeuser{r1}{ 0,0}{Reinigung 1};
   \activeuser{r2}{ 2,0}{Reinigung 2};	
   \draw[dotted] (3,4) -- (3,-1);
   \user{s1}{ 4,0}{Sicherheit 1};
   \activeuser{z} { 6,0}{Zentrale};
   \draw[dotted] (7,4) -- (7,-1);
   \user{s2}{ 8,0}{Sicherheit 2};
   \user{r3}{10,0}{Reinigung 3};
   \activerepeater{R1}{1,3}{Terminal 1, TG: R,S};
   \activerepeater{R2}{5,3}{Terminal 2, TG: R,S};
   \repeater{R3}{9,3}{Vorfeld, TG: S};
   \draw[->] (z) -- node[above,rotate=-74] {TG: R} (R2);
   \path[->,dashed] (R2) edge [bend right] node[above] {via Netzwerk} (R1);
   \draw[->] (R1) -- node[above,rotate=74] {TG: R} (r1);
   \draw[->] (R1) -- node[above,rotate=-74] {TG: R} (r2);
  \end{tikzpicture}
\end{document}

  \caption{Ein Gruppenruf zur Sprechgruppe \emph{Reinigung} von der Zentrale aus. Der Teilnehmer \emph{Reinigung 3} wird aber nicht erreicht, da der Vorfeldrepeater diese Sprechgruppe nicht abonniert hat.} \label{fig:exnet3}
 \end{subfigure}\vspace{0.5cm}
 \begin{subfigure}{\linewidth}
  \centering
  \documentclass{standalone}
\usepackage{tikz}
\usetikzlibrary{shapes.geometric}
\newcommand{\repeater}[3]{%
 \node ({#1}) at ({#2}) {%
  \begin{tikzpicture}%
   \draw [black,thick] (-.25,0) -- (0,0.5) -- (0.25,0) -- (-0.25,0);%
   \draw [black,thick,domain=-45:225] plot ({0.2*cos(\x)}, {0.5+0.2*sin(\x)});%
   \draw [black,thick,domain=-45:225] plot ({0.4*cos(\x)}, {0.5+0.4*sin(\x)});%
   \node (xxx) at (0,-.2) {{#3}};%
  \end{tikzpicture}%
 } %
}

\newcommand{\activerepeater}[3]{%
 \node ({#1}) at ({#2}) {%
  \begin{tikzpicture}%
   \draw [black,thick] (-.25,0) -- (0,0.5) -- (0.25,0) -- (-0.25,0);%
   \draw [red,thick,domain=-45:225] plot ({0.2*cos(\x)}, {0.5+0.2*sin(\x)});%
   \draw [red,thick,domain=-45:225] plot ({0.4*cos(\x)}, {0.5+0.4*sin(\x)});%
   \node (xxx) at (0,-.2) {{#3}};%
  \end{tikzpicture}%
 } %
}


\newcommand{\user}[3]{%
 \node ({#1}) at ({#2}) {%
  \begin{tikzpicture}%
   \draw [black,fill=black] (-.25,0) -- (0,0.5) -- (0.25,0) -- (-0.25,0);%
   \draw [black,fill=black] (0,.5) circle (.2); %
   \node (xxx) [text width=0.6cm, align=center] at (-.35cm,-.4) {{#3}};%
  \end{tikzpicture}%
 } %
}

\newcommand{\activeuser}[3]{%
 \node ({#1}) at ({#2}) {%
  \begin{tikzpicture}%
   \draw [red,fill=red] (-.25,0) -- (0,0.5) -- (0.25,0) -- (-0.25,0);%
   \draw [red,fill=red] (0,.5) circle (.2); %
   \node (xxx) [text width=0.6cm, align=center] at (-.35cm,-.4) {{#3}};%
  \end{tikzpicture}%
 } %
}

\begin{document}
  \begin{tikzpicture}[every node/.style={scale=.8}]
   \activeuser{r1}{ 0,0}{Reinigung 1};
   \activeuser{r2}{ 2,0}{Reinigung 2};	
   \draw[dotted] (3,4) -- (3,-1);
   \user{s1}{ 4,0}{Sicherheit 1};
   \user{z} { 6,0}{Zentrale};
   \draw[dotted] (7,4) -- (7,-1);
   \user{s2}{ 8,0}{Sicherheit 2};
   \user{r3}{10,0}{Reinigung 3};
   \activerepeater{R1}{1,3}{Terminal 1, TG: R,S};
   \repeater{R2}{5,3}{Terminal 2, TG: R,S};
   \activerepeater{R3}{9,3}{Vorfeld, TG: S,(R)};
   \draw[->] (r3) -- node[above,rotate=-74] {TG: R} (R3);
   \path[->,dashed] (R3) edge [bend right] node[above] {via Netzwerk} (R1);
   \draw[->] (R1) -- node[above,rotate=74] {TG: R} (r1);
   \draw[->] (R1) -- node[above,rotate=-74] {TG: R} (r2);
  \end{tikzpicture}
\end{document}

  \caption{Teilnehmer \emph{Reinigung 3} abonniert die Sprechgruppe \emph{Reinigung} temporär auf dem Vorfeldrepeater, indem er einen Gruppenruf zu dieser Sprechgruppe startet.} \label{fig:exnet4a} 
 \end{subfigure}\vspace{.5cm}
 \begin{subfigure}{\linewidth}
  \centering
  \documentclass{standalone}
\usepackage{tikz}
\usetikzlibrary{shapes.geometric}
\newcommand{\repeater}[3]{%
 \node ({#1}) at ({#2}) {%
  \begin{tikzpicture}%
   \draw [black,thick] (-.25,0) -- (0,0.5) -- (0.25,0) -- (-0.25,0);%
   \draw [black,thick,domain=-45:225] plot ({0.2*cos(\x)}, {0.5+0.2*sin(\x)});%
   \draw [black,thick,domain=-45:225] plot ({0.4*cos(\x)}, {0.5+0.4*sin(\x)});%
   \node (xxx) at (0,-.2) {{#3}};%
  \end{tikzpicture}%
 } %
}

\newcommand{\activerepeater}[3]{%
 \node ({#1}) at ({#2}) {%
  \begin{tikzpicture}%
   \draw [black,thick] (-.25,0) -- (0,0.5) -- (0.25,0) -- (-0.25,0);%
   \draw [red,thick,domain=-45:225] plot ({0.2*cos(\x)}, {0.5+0.2*sin(\x)});%
   \draw [red,thick,domain=-45:225] plot ({0.4*cos(\x)}, {0.5+0.4*sin(\x)});%
   \node (xxx) at (0,-.2) {{#3}};%
  \end{tikzpicture}%
 } %
}


\newcommand{\user}[3]{%
 \node ({#1}) at ({#2}) {%
  \begin{tikzpicture}%
   \draw [black,fill=black] (-.25,0) -- (0,0.5) -- (0.25,0) -- (-0.25,0);%
   \draw [black,fill=black] (0,.5) circle (.2); %
   \node (xxx) [text width=0.6cm, align=center] at (-.35cm,-.4) {{#3}};%
  \end{tikzpicture}%
 } %
}

\newcommand{\activeuser}[3]{%
 \node ({#1}) at ({#2}) {%
  \begin{tikzpicture}%
   \draw [red,fill=red] (-.25,0) -- (0,0.5) -- (0.25,0) -- (-0.25,0);%
   \draw [red,fill=red] (0,.5) circle (.2); %
   \node (xxx) [text width=0.6cm, align=center] at (-.35cm,-.4) {{#3}};%
  \end{tikzpicture}%
 } %
}

\begin{document}
  \begin{tikzpicture}[every node/.style={scale=.8}]
   \activeuser{r1}{ 0,0}{Clean 1};
   \activeuser{r2}{ 2,0}{Clean 2};	
   \draw[dotted] (3,4) -- (3,-1);
   \user{s1}{ 4,0}{Security 1};
   \activeuser{z} { 6,0}{HQ};
   \draw[dotted] (7,4) -- (7,-1);
   \user{s2}{ 8,0}{Security 2};
   \activeuser{r3}{10,0}{Clean 3};
   \activerepeater{R1}{1,3}{Terminal 1, TG: C,S};
   \activerepeater{R2}{5,3}{Terminal 2, TG: C,S};
   \activerepeater{R3}{9,3}{Apron, TG: S,(C)};
   \draw[->] (z) -- node[above,rotate=-74] {TG: C} (R2);
   \path[->,dashed] (R2) edge [bend right] node[above] {via Netzwerk} (R1);
   \path[->,dashed] (R2) edge [bend left] node[above] {via Netzwerk} (R3);
   \draw[->] (R1) -- node[above,rotate=74] {TG: C} (r1);
   \draw[->] (R1) -- node[above,rotate=-74] {TG: C} (r2);
   \draw[->] (R3) -- node[above,rotate=-74] {TG: C} (r3);
  \end{tikzpicture}
\end{document}

  \caption{Nach der temporären Abonnierung, ist nun der Teilnehmer \emph{Reinigung 3} auch auf dem Vorfeld erreichbar.} \label{fig:exnet4b}
 \end{subfigure}
 \caption{Temporäres Abonnement einer Sprechgruppe auf einem Repeater.} \label{fig:exnet4}
\end{figure}

Damit die Reinigungskraft 3 jedoch für Gruppenrufe erreichbar bleibt muss sie die Sprechgruppe \emph{Reinigung} auf dem Vorfeldrepeater temporär abonnieren. Dazu startet sie einen Gruppenruf zur Sprechgruppe \emph{Reinigung} vom Vorfeldrepeater aus (siehe Abb. \ref{fig:exnet4a}). Damit abonniert der Vorfeldrepeater diese Sprechgruppe für eine begrenzte Zeit\footnote{Diese Zeit wird auf jedem einzelnen Repeater konfiguriert. Üblich sind Zeiten zwischen $10$ und $30$ Minuten.} und wird während dieser Zeit Gruppenrufe dieser Sprechgruppe aussenden. 

Dieses temporäre Abonnement wird jedes mal erneuert oder wiederhergestellt, wenn ein Gruppenruf zu dieser Sprechgruppe von diesem Repeater initiiert wird. Das heißt, das Abonnement verlängert sich jedes mal, wenn \emph{Reinigung 3} einen Gruppenruf zur Sprechgruppe \emph{Reinigung} startet oder darauf antwortet\footnote{Das Antworten auf einen Gruppenruf ist technisch identisch zum Start eines neuen Gruppenrufs.}.

Mit diesen Beispielen sind die wichtigsten Grundbegriffe von DMR (DMR-ID, Talk Groups, Private sowie Group Call, Talk Group Abonnement) eingeführt und deren Verwendung in einem Beispiel DMR Netz erläutert worden. In den nächsten Absetzen wird die Verwendung von DMR im Amateurfunk beschrieben.



\section{DMR Simplex Betrieb} \label{sec:simplex}
Die einfachste Form eines DMR QSOs\footnote{Für alle nicht-Funkamateure: QSO ist eine Abkürzung die eine Verbindung zwischen zwei Amateurfunkstationen beschreibt, gelesen als \emph{Verbindung} oder \emph{Gespräch}.} ist der \aref{Simplexbetrieb}. Dabei wird eine direkte Verbindung zwischen zwei DMR Funkgeräten aufgebaut, die sich beide direkt erreichen können. Wie bei dem DMR Repeaterbetrieb, kann so eine Verbindung ein Direktruf, Gruppenruf oder auch ein sog. \adef{All Call} sein. 

\begin{figure}[!ht]
 \centering
 \begin{tikzpicture}[every node/.style={scale=.8}]
  \activeuser{u1}{ 0,0}{DM3MAT};
  \activeuser{u2}{ 6,1}{DL1XYZ, TG99};
  \user{u3}{ 6,0}{DL2XYZ, TG99};
  \user{u4}{ 6,-1}{DL3XYZ};
  \path[->] (u1) edge[bend left] node[above, rotate=10]{$433.450 MHz$} node[below, rotate=10]{PC: DL1XYZ} (u2);
  \path[->] (u2) edge[bend left] node[above, rotate=10]{$433.450 MHz$} node[below, rotate=10]{PC: DM3MAT} (u1);
 \end{tikzpicture}
 \caption{Beispiel eines DMR Simplex Direktrufs von DM3MAT an DL1XYZ.} \label{fig:splxpc}
\end{figure}

In Abbildung \ref{fig:splxpc} ist ein einfache Simplex Direktruf von DM3MAT an DL1XYZ dargestellt sowie die Antwort von DL1XYZ and DM3MAT. Beide senden und empfangen auf der selben Frequenz (hier der DMR Simplex Anruffrequenz von $433.450 MHz$). Auch wenn die beiden anderen Teilnehmer in der Nähe (DL2XYZ \& DL3XYZ) diesen Ruf physisch empfangen, bleiben deren Funkgeräte stumm. Wie dem auch sei, der Kanal ist jedoch während dieses Direktrufes belegt. 

An dieser Stelle ist es sinnvoll zu erwähnen, dass wenn DL1XYZ direkt auf den Direktruf von DM3MAT antwortet indem er die PTT Taste drückt, er mit einem Direktruf an DM3MAT antwortet ohne dafür die Nummer von DM3MAT aus seinen Kontakten heraussuchen zu müssen. Diese Eigenschaft heißt \adef{Talkaround} und funktioniert nur wenige Sekunden nach dem Ende des initialen Direktrufs durch DM3MAT. Nach dieser Zeitspanne wird beim drücken auf die PTT der Standardkontakt für diesen Kanal angerufen, der für jeden Kanal im Funkgerät festgelegt werden kann (siehe Abs. \ref{sec:cp:channel}). Diese Zeitspanne lässt sich auch im Funkgerät einstellen.

\begin{figure}[!ht]
  \centering
  \begin{tikzpicture}[every node/.style={scale=.8}]
   \activeuser{u1}{ 0,0}{DM3MAT};
   \user{u2}{ 6,1}{DL1XYZ, TG99};
   \user{u3}{ 6,0}{DL2XYZ, TG99};
   \user{u4}{ 6,-1}{DL3XYZ};
   \path[->] (u1) edge[bend left] node[above, rotate=10]{$433.450 MHz$} node[below, rotate=10]{GC: TG99} (u2);
   \path[->] (u1) edge node[above]{$433.450 MHz$} node[below]{GC: TG99} (u3);
  \end{tikzpicture}
  \caption{Beispiel eines DMR Simplex Gruppenrufs von DM3MAT an die Sprechgruppe TG99.} \label{fig:splxgc}
\end{figure}

Um im Simplexbetrieb nicht nur einzelne Teilnehmer anrufen zu können, sind auch Gruppenrufe im Simplexbetrieb möglich. Eine beliebte Sprechgruppe (Talk Group) für den Simplex Betrieb ist die Gruppe mit der Nummer 99, daher mit TG99 abgekürzt für \emph{talk group 99}. Solche Gruppenrufe werden dann von allen Funkgeräten empfangen, die entsprechend konfiguriert wurden. Wie beim Repeaterbetrieb muss auch beim Simplexbetrieb dem Funkgerät mitgeteilt werden, welche Sprechgruppen es auf welchen Kanälen empfangen soll (siehe Abs. \ref{sec:cp:rxgrplst}). 

In Abbildung \ref{fig:splxgc} ist solch ein Simplex Gruppenruf von DM3MAT an die Sprechgruppe TG99 dargestellt. Da DL1XYZ und DL2XYZ ihre Funkgeräte so konfiguriert haben, dass sie die TG99 empfangen, hören sie den Ruf von DM3MAT. Da DL3XYZ dies nicht gemacht hat, empfängt er diesen Ruf nicht. DL1XYZ und DL2XYZ können nun auf diesen Gruppenruf antworten, wenn sie innerhalb der sog. Hangtime auf ihre PTT Taste drücken. Sie würden dann ebenfalls mit einem Gruppenruf zur TG99 antworten (Talkaround gilt auch für Gruppenrufe), auch wenn sie einen anderen Standardkontakt für diesen Simplexkanal eingestellt haben.

\begin{figure}[!ht]
  \centering
  \begin{tikzpicture}[every node/.style={scale=.8}]
   \activeuser{u1}{ 0,0}{DM3MAT};
   \user{u2}{ 6,1}{DL1XYZ, TG99};
   \user{u3}{ 6,0}{DL2XYZ, TG99};
   \user{u4}{ 6,-1}{DL3XYZ};
   \path[->] (u1) edge[bend left] node[above, rotate=10]{$433.450 MHz$} node[below, rotate=10]{All Call} (u2);
   \path[->] (u1) edge node[above]{$433.450 MHz$} node[below]{GC: TG99} (u3);
   \path[->] (u1) edge[bend right] node[above, rotate=-10]{$433.450 MHz$} node[below, rotate=-10]{All Call} (u4);
  \end{tikzpicture}
  \caption{Beispiel eines DMR All Calls von DM3MAT alle die ihn hören können.} \label{fig:splxac}
\end{figure}

Um wirklich sicher zu gehen, dass ein Ruf auf einem Simplexkanal von allen empfangen werden kann, sollte ein sogenannter \adef{All Call} verwendet werden. Dieser Ruf ist ein spezieller Ruf an eine ganz bestimmte Nummer ($16777215$), die von allen Geräten empfangen werden unabhängig von der Konfiguration der Geräte. In diesem Beispiel wird somit der Ruf von DM3MAT auch von DL3XYU empfangen. Durch \emph{Talkaround} ist es allen Teilnehmern möglich auf den Ruf von DM3MAT zu antworten, auch wenn diese Teilnehmer den All-Call nicht als den Standardkontakt für diesen Kanal konfiguriert haben. 



\section{Direkte Anrufe} \label{sec:privatecall}
Direktrufe (\aref{Private Call}) ermöglichen es mit einem anderen Teilnehmer direkt zu kommunizieren, ohne dabei weitere Teilnehmer zu stören (bis auf das Belegen eines Repeaters). Im Rahmen der DMR Einführung wurde der Direktruf auch über mehrere Repeater hinweg beschrieben. Eben dieser Aspekt von DMR ist meiner Meinung nach besonders interessant. Mit Ausnahme der Sprechgruppen 8 \& 9 (siehe Abs. \ref{sec:lokal}), sind Direkt- und Gruppenrufe in DMR transparent gegenüber den verwendeten Repeatern. Es spielt keine Rolle über welchen Repeater sich Teilnehmer an einem Direkt- oder Gruppenruf beteiligen und somit auch nicht wo sie sich befinden. 

\begin{figure}[!ht]
 \centering
 \documentclass{standalone}
\usepackage{tikz}
\usetikzlibrary{shapes.geometric}
\newcommand{\repeater}[3]{%
 \node ({#1}) at ({#2}) {%
  \begin{tikzpicture}%
   \draw [black,thick] (-.25,0) -- (0,0.5) -- (0.25,0) -- (-0.25,0);%
   \draw [black,thick,domain=-45:225] plot ({0.2*cos(\x)}, {0.5+0.2*sin(\x)});%
   \draw [black,thick,domain=-45:225] plot ({0.4*cos(\x)}, {0.5+0.4*sin(\x)});%
   \node (xxx) at (0,-.2) {{#3}};%
  \end{tikzpicture}%
 } %
}

\newcommand{\activerepeater}[3]{%
 \node ({#1}) at ({#2}) {%
  \begin{tikzpicture}%
   \draw [black,thick] (-.25,0) -- (0,0.5) -- (0.25,0) -- (-0.25,0);%
   \draw [red,thick,domain=-45:225] plot ({0.2*cos(\x)}, {0.5+0.2*sin(\x)});%
   \draw [red,thick,domain=-45:225] plot ({0.4*cos(\x)}, {0.5+0.4*sin(\x)});%
   \node (xxx) at (0,-.2) {{#3}};%
  \end{tikzpicture}%
 } %
}


\newcommand{\user}[3]{%
 \node ({#1}) at ({#2}) {%
  \begin{tikzpicture}%
   \draw [black,fill=black] (-.25,0) -- (0,0.5) -- (0.25,0) -- (-0.25,0);%
   \draw [black,fill=black] (0,.5) circle (.2); %
   \node (xxx) [text width=0.6cm, align=center] at (-.35cm,-.4) {{#3}};%
  \end{tikzpicture}%
 } %
}

\newcommand{\activeuser}[3]{%
 \node ({#1}) at ({#2}) {%
  \begin{tikzpicture}%
   \draw [red,fill=red] (-.25,0) -- (0,0.5) -- (0.25,0) -- (-0.25,0);%
   \draw [red,fill=red] (0,.5) circle (.2); %
   \node (xxx) [text width=0.6cm, align=center] at (-.35cm,-.4) {{#3}};%
  \end{tikzpicture}%
 } %
}

\begin{document}
 \begin{tikzpicture}[every node/.style={scale=.8}]
  \activeuser{u1}{ 0,0}{DM3MAT 2621370};
  \activerepeater{R1}{1,3}{DB0ABC};
  \draw[dotted] (2,4) -- (2,-1);
  \user{u2}{ 4,0}{I/DL2XYZ\\2621234};	
  \activerepeater{R2}{3,3}{I0ABC};
  \draw[->] (u1) -- node[above,rotate=70]{PC: 2621234} (R1);
  \draw[->] (R2) -- node[above,rotate=-70]{PC: 2621234} (u2);
  \path[->] (R1) edge[dashed,bend left] node[above]{via network} (R2);
 \end{tikzpicture}
\end{document}

 \caption{Beispiel eines Direktrufs über Ländergrenzen hinweg.} \label{fig:pc}
\end{figure}

D.h., YLs \& OMs \footnote{Für nicht-Funkamateure: Zwei weitere typische Abkürzungen im Amateurfunk die \emph{young lady} und \emph{old man} bedeuten und alle weiblichen b.z.w. männlichen Funkamateure bezeichnet.}, die sich im Urlaub aufhalten können wie gewohnt an ihren lokalen Runden teilnehmen, indem sie am Urlaubsort einen DMR Repeater auswählen und von dort aus einen Gruppenruf zu ihrer Sprechgruppe in der Heimat starten. Damit abonnieren sie ihre Sprechgruppe an ihrem Urlaubs Repeater temporär und dieser verhält sich danach wie ein Repeater der in der Heimat steht. 

Ebenso können sie auch Direktrufe vom Urlaubsort an Bekannte absetzen und auch am Urlaubsort empfangen. Vorausgesetzt, sie haben sich durch kurzes drücken auf die PTT Taste beim Repeater am Urlaubsort angemeldet damit das DMR Netz weiß wo der Teilnehmer zu finden ist. Damit müssen die Teilnehmer in der Heimat aber nicht mehr wissen wie und wo sie den Urlauber erreichen können. Sie starten einfach einen Direktruf zum Urlauber und das DMR Netz kümmert sich um alles.

In Abbildung \ref{fig:pc} ist eben solch ein Direktruf über Ländergrenzen hinweg dargestellt. DM3MAT ruft via seinem lokalen Repeater (DB0ABC) DL2XYZ per Direktruf an. Da sich dieser bei einem DMR Repeater (I0ABC) an seinem Urlaubsort in Italien angemeldet hat\footnote{Um sich an einem Repeater anzumelden, damit das Netzwerk weiß, dass man über diesen Repeater erreichbar ist, drückt man kurz die PTT Taste auf einem Kanal des Repeaters.}, kann der Direktruf an DL2XYZ vermittelt werden. Um diesen Direktruf durchzuführen, muss DM3MAT nicht wissen über welchen Repeater DL2XYZ erreichbar ist. Diese Eigenschaft des DMR Netzes stellt eine deutliche Vereinfachung gegenüber dem \aref{Echolink} Netzwerk dar. 

\section{Lokaler Repeater Betrieb} \label{sec:lokal}	

\section{Talkgroup Betrieb} \label{sec:talkgroup}

\section{Datendienste} \label{sec:data}
Da DMR von sich aus schon eine digitale Betriebsart ist, bei der meist Sprache in digitalisierter 
Form übertragen wird, ist es natürlich auch möglich reine Datendienste über DMR anzubieten. Zum 
einen gibt es einen Textnachrichtendienst, der dem SMS-Dienst der Mobiltelefone nachempfunden ist. 
Zum anderen gibt es auch die Möglichkeit, die eigene Position per DMR an das APRS\footnote{APRS 
steht für \emph{Automatic Packet Reporting System} und ermöglicht das Übertragen von kleinen 
Datensätzen über Packet-Radio wie zum Beispiel die Position, Wetter oder Textnachrichten. Mehr 
dazu erfahren sie in der \href{https://de.wikipedia.org/wiki/Automatic_Packet_Reporting_System}{Wikipedia}.} 
Netz zu übertragen.
 
\subsection{Textnachrichten (SMS)} \label{sec:textmsg}
Mit diesem Dienst können sie kurze Textnachrichten\footnote{Bis zu 144 Zeichen.} direkt an andere Teilnehmer verschicken\footnote{Sie können auch Textnachrichten an ganze Sprechgruppen versenden. Dies ist aber eher unüblich und nicht wünschenswert.}. Im Prinzip funktioniert eine Textnachricht wie ein Direktruf. Ist der andere Teilnehmer erreichbar, wird die Textnachricht übermittelt. 

Es gibt aber auch \emph{Servicenummern} (gebührenfrei). Wenn sie nun eine Nachricht an eine solche Nummer senden, können Sie bestimmte Informationen abrufen oder versenden. In Deutschland wären das:
\begin{enumerate}
 \item 262993 -- GPS und Wetter
 \begin{itemize}
  \item Wenn Sie \texttt{help} senden, erhalten daraufhin eine Auflistung aller Kommandos.
  \item Wenn Sie \texttt{wx} senden, erhalten Sie das aktuelle Wetter am Standort des Repeaters, den Sie verwenden.
  \item Wenn Sie \texttt{wx STADTNAME} senden, erhalten Sie das aktuelle Wetter für die angegebene Stadt.
  \item Wenn Sie \texttt{gps} senden, erhalten Sie die letzte Positionsinformation, die Sie zuletzt an das DMR Netz gesendet hatten.
  \item Mit \texttt{gps CALL} können Sie auch die letzte Position des angegebenen Teilnehmers abfragen.
  \item Mit \texttt{rssi} erhalten Sie vom Repeater einen Signalrapport.
 \end{itemize} 
 \item 262994 -- Repeater Informationen \& Pagernachrichten
  \begin{itemize}
   \item Wenn Sie \texttt{rpt} senden, erhalten Sie eine Liste der statisch und dynamisch abonnierten Sprechgruppen des Repeaters.  
   \item Wenn Sie \texttt{CALL NACHRICHT}, wird die angegebene Nachricht an das angegebene Call per Pager (DAPNET) geschickt.
  \end{itemize}
\end{enumerate}
 
\subsection{Positionsübermittlung (APRS via DMR)} \label{sec:aprs}
Wie im vorherigen Abschnitt schon erwähnt, ist es möglich seine Position ins DMR Netz zu senden. Diese wird dann üblicherweise direkt an das APRS-Netz weitergereicht und Ihre Position kann dann unter anderem bei \url{https://aprs.fi} abgefragt werden. Dazu ist jedoch ein DMR Funkgerät mit GPS Empfänger nötig. Aber auch diese Geräte sind in der Zwischenzeit nicht mehr teuer. Einfache DMR Handfunkgeräte mit GPS sind ab circa \EUR{120} zu haben. 

Neben dem SMS Service ist auch die Positionsübermitellung per DMR möglich. Dazu muss das GPS fähige Funkgerät so konfiguriert werden, dass die Positionsdaten auf den geeigneten Kanälen an die Nummer 262999 gesendet werden. Wie dies einzustellen ist, hängt sehr vom Hersteller des Funkgerätes ab. 

\section{Roaming} \label{sec:roaming}

\section{Technischer Hintergrund} \label{sec:technik}

\section{Codeplug Programmierung} \label{sec:codeplug}
Nachdem Sie sich mit den Konzepten und dem technischen Hintergrund von DMR auseinandergesetzt haben, geht es nun an die Konfiguration Ihres Funkgerätes. Dies geschieht üblicherweise nicht über das Bedienfeld des Funkgerätes, sonder mit Hilfe einer separaten Software, der sogenannten \adef{CPS} oder \emph{codeplug programming software}. 

Doch bevor Sie loslegen können benötigen Sie wie alle DMR Teilnehmer eine eindeutige Nummer, die DMR ID.
\begin{hinweis}
 Ihre persönliche und eindeutige DMR ID erhalten Sie unter \url{https://register.ham-digital.org/}. Da Sie nachweisen müssen, dass Sie lizenzierter Funkamateur sind, müssen Sie bei der Anmeldung ihre eingescannte \emph{Zulassung zum Amateurfunkdienst} hochladen.
\end{hinweis}
Ihre DMR ID erhalten Sie in der Regel innerhalb von 24 Stunden per Mail. Sobald Sie eine DMR ID erhalten haben kann es los gehen.

Da dieses Script für Einsteiger gedacht ist, ist es wahrscheinlich, dass Sie kein top-shelf Motorola Gerät sondern eher ein günstiges Gerät der einschlägig bekannten chinesischen Hersteller besitzen. 

\begin{achtung}
 Falls Sie noch kein DMR fähiges Funkgerät besitzen und mit dem Gedanken spielen eines zu kaufen, achten Sie unbedingt darauf, dass es DMR \textbf{Tier I \& II}\footnote{Wie so häufig ist DMR nicht ein Standard sondern eine ganze Familie von aufeinander aufbauenden Standards. DMR Tier I beschreibt im wesentlichen den DMR Simplexbetrieb und Tier II dann den Repeaterbetrieb mit zwei Zeitschlitzen. Sie benötigen also unbedingt Tier II für den Repeaterbetrieb.} unterstützt. Ignorieren Sie etwaiges Marketing-Bla-Bla der Hersteller und schauen Sie in den technischen Details nach, ob dort DMR \textbf{Tier I \& II} erwähnt wird. Falls nicht oder nicht eindeutig, lassen Sie die Finger von diesem Gerät! Dies gilt vor allem für das Baofeng MD-5R aber nicht für das Baofeng/Radioddity RD-5R\footnote{Manchmal sind es die kleinen Unterschiede die entscheidend sind.}. 
\end{achtung}

Der Hersteller Ihres Gerätes wird auf seiner Webseite die Software die Sie zur Konfiguration benötigen, zum Download bereitstellen. Diese Software wird \emph{CPS} oder \emph{codeplug programming software} genannt. Gegebenenfalls finden Sie dort auch Firmwareupdates für Ihr Gerät. Viele Hersteller bieten für jedes einzelne Modell eine separate CPS an oder gar für jede Variation eines Modells. Achten Sie also genau darauf welche CPS Sie herunterladen. Die Konfiguration dieser Geräte unterscheidet sich von Gerät zu Gerät und mehr noch von Hersteller zu Hersteller. Jedoch sind die wesentlichen Einstellungen für Geräte dieser Klasse sehr ähnlich.

Wenn Sie die CPS zum ersten mal starten, werden Sie wahrscheinlich zwei Dinge feststellen. Erstens, das Bedienkonzept dieser Software ist aus dem letzten Jahrtausend (Windows 3.11) und Zweitens, es gibt eine Unmenge an obskuren Optionen deren Funktion nicht ersichtlich ist und die größtenteils nicht Dokumentiert sind. Wenn Sie des Englischen nicht mächtig sind, werden Sie auch eine deutsche Übersetzung des Programms vermissen. Aber keine sorge, die englische Übersetzung ist meist auch so schlecht, dass es keinen Unterschied macht ob sie Englisch lesen können oder nicht.

Die Konfiguration Ihres Funkgerätes erfolgt in 5-6 Schritten:
\begin{enumerate}
 \item Allgemeine Einstellungen,
 \item Kontakte anlegen,
 \item Empfangsgruppen festlegen,
 \item alle Kanäle anlegen,
 \item Kanäle in Zonen einteilen und
 \item optional Scanlisten anlegen.
\end{enumerate}

In den folgenden Abschnitten möchte ich die einzelnen Konfigurationsschritte im Detail beschreiben.

\subsection{Allgemeine Konfiguration} \label{sec:cp:basic}
Die wichtigsten allgemeinen Einstellungen die Sie vornehmen müssen, ist das setzen der DMR ID und ihres Rufzeichens. Diese Optionen finden Sie meist unter der Rubrik (linke Seite) \emph{Radio Settings} oder \emph{General Settings}\footnote{Die exakten Namen der Rubriken und Felder kann sich von Hersteller zu Hersteller unterscheiden. Üblicherweise sind sie aber den hier erwähnten Namen sehr ähnlich.}. Ihre DMR ID tragen Sie dann in das Feld \emph{Radio ID} ein. Es ist durchaus möglich, dass Ihr Funkgerät mehrere DMR IDs unterstützt. Dieses Feature wird aber üblicherweise nicht verwendet. Im Gegenteil: Es stehen nur eine begrenzte Anzahl von DMR IDs sehr vielen Funkamateuren gegenüber. Beantragen sie deshalb niemals eine DMR ID für jedes Funkgerät oder jeden Accesspoint. Eine \textbf{einzige} persönliche DMR Nummer reicht völlig! 

Ihr Rufzeichen tragen Sie bitte in das Feld \emph{Radio Name} ebenfalls in der Rubrik \emph{Radio Settings} ein.  

\subsection{Kontakte Anlegen} \label{sec:cp:contact}
Nachdem Sie die grundlegenden Einstellungen vorgenommen haben, können Sie Ihre Kontaktliste zusammenstellen. Diese sollte alle Sprechgruppen enthalten die Sie interessieren könnten, ihre persönlichen Kontakte wie OMs aus dem OV und einige Servicenummern wie Echo, die SMS Dienste und den All Call. Eine Beispiel für Deutschland ist in Tabelle \ref{tab:contacts} angegeben.

\begin{table}[!ht]
 \centering
 \begin{tabular}{|l|c|c||l|c|c|}\hline
  Name        & Typ        & Nummer & Name & Typ & Nummer \\ \hline
  Lokal       & Gruppenruf & 9        & Ham/SlHo    & Gruppenruf & 2622 \\
  Regional    & Gruppenruf & 8        & NiSa/Bre    & Gruppenruf & 2623 \\
  TG99        & Gruppenruf & 99       & NRW         & Gruppenruf & 2624 \\
  Rundumruf   & All Call   & 16777215 & RhPf/Saar   & Gruppenruf & 2625 \\
  Weltweit    & Gruppenruf & 91       & Hessen      & Gruppenruf & 2626 \\
  Europa      & Gruppenruf & 92       & BaWü        & Gruppenruf & 2627 \\
  D-A-CH      & Gruppenruf & 920      & Bay         & Gruppenruf & 2628 \\
  Deutschland & Gruppenruf & 262      & Sa/Th       & Gruppenruf & 2629 \\
  Österreich  & Gruppenruf & 232      & Echo Test   & Direktruf  & 262997 \\
  Schweiz     & Gruppenruf & 228      & SMS Serv.   & Direktruf  & 262993 \\
  EMCOM\footnote{Ausschließlich für Notfunk.} EU    & Gruppenruf & 9112 & 
  DAPNET      & Direktruf  & 262994 \\
  EMCOM WW    & Gruppenruf & 9911     & APRS GW     & Direktruf  & 262999 \\
  MeVo/SaAn   & Gruppenruf & 2620     & DM3MAT      & Direktruf  & 2621370 \\
  Ber/Bra     & Gruppenruf & 2621     & ...         & ...        & ... \\ \hline
 \end{tabular}
 \caption{Beispielkontakte für Deutschland.} \label{tab:contacts}
\end{table}

Natürlich gibt es noch viele weitere Sprechgruppen auch zu spezifischen Themen, die nicht unbedingt regional beschränkt sind. Eine recht vollständige Liste finden Sie unter \url{https://www.pistar.uk/dmr_bm_talkgroups.php}.


\subsection{Empfangsgruppen Zusammenstellen} \label{sec:cp:grouplist}
Im nächsten Schritt stellen Sie sogenannte \adef{Empfangsgruppen} zusammen. Dies sind Listen von Gruppenrufen, die Sie auf bestimmten Kanälen empfangen wollen. Wie schon bei der Einführung in Abschnitt \ref{sec:ursprung} erwähnt, weiß das DMR Netz nicht, für welche Sprechgruppen Sie sich interessieren. Dies kann nur Ihr Funkgerät wissen. Mit den Empfangsgruppen definieren Sie genau das. Sie werden mindestens drei Empfangsgruppen benötigen. Eine für den Simplexbetrieb, eine für die überregionale Kommunikation und je eine für regionale Kommunikation in all jenen Regionen, in denen Sie unterwegs sind.

Die Simplex Empfangsgruppe ist eigentlich nicht notwendig, da Simplexrufe eigentlich immer den sog. \aref{All Call} (Rundumruf) verwenden sollten. Häufig wird aber auch die Sprechgruppe TG99, TG9 oder auch TG8 verwendet. Daher ist es ratsam eine Empfangsgruppe mit diesen Gruppenrufen anzulegen. 

Für die überregionale Kommunikation sollte eine Empfangsgruppe erstellt werden, die die Sprechgruppen für weltweite, innereuropäische und deutschlandweite Kommunikation enthalten. Dieser Gruppe können Sie dann noch die Sprechgruppe \emph{EMCOM EU} für europäischen Notfunk hinzufügen, damit Sie ggf. Notrufe hören und darauf reagieren können. 

Zuletzt sollte die Sprechgruppen für lokale/regionale Kommunikation angelegt werde. Diese sollte jeweils die Sprechgruppen TG8 und TG9 sowie die Sprechgruppe der jeweiligen Region enthalten. Für mich, der in der Berlin/Brandenburg Region lebt, aber häufig auch in Sachsen unterwegs ist, habe ich insgesamt 4 Empfangsgruppen zusammengestellt (siehe Tab. \ref{tab:grouplist}).
  
\begin{table}
 \centering
 \begin{tabular}{|l|l|} \hline
 Name     & Gruppenrufe \\ \hline
 Simplex  & Lokal, Regional, TG99 \\
 WW/EU/DL & Weltweit, Europa, D-A-CH, Deutschland, EMCOM EU \\
 Ber/Bra  & Lokal, Regional, Ber/Bra \\
 Sa/Th    & Lokal, Regional, Sa/Th \\ \hline
 \end{tabular}
 \caption{Ein paar Beispielempfangsgruppen. Die ersten beiden sind recht universell für Deutschland, die letzten Beiden sind für die Regionen Berlin/Brandenburg und Sachsen/Thüringen wichtig.} \label{tab:grouplist}
\end{table}


\subsection{Kanäle Anlegen} \label{sec:cp:channel}
Bevor es los geht, sollte ich erwähnen, dass die meisten DMR Funkgeräte auch analoges FM unterstützen. Das heißt, Sie können mit ihrem DMR Funkgerät auch normalen analogen FM Simplex und Repeaterbetrieb durchführen. In diesem Abschnitt beschreibe ich aber nur die Konfiguration von DMR Kanälen (meist \emph{Digital Channel} genannt), die Konfiguration von sogenannten \emph{analogen} Kanälen wird hier nicht beschrieben. Um einen DMR Kanal anzulegen, müssen Sie im Feld \emph{Channel Type} den Wert \emph{digital} auswählen, für einen FM Kanal dann \emph{analog}.

Wenn Sie schon Erfahrung mit dem \emph{klassischen} FM-Relaisbetrieb haben, wird Ihnen das Anlegen der Kanäle recht seltsam vorkommen. Im analogen FM-Relaisbetrieb haben Sie für jeden Repeater und Simplex-Kanal genau einen Kanal im Funkgerät konfiguriert. Für den DMR Betrieb werden Sie für jeden Repeater mindestens zwei (für Zeitschlitz 1 \& 2), meist aber deutlich mehr Kanäle programmieren. Lange Rede kurzer Sinn. Lassen Sie mich das an konkreten Beispielen erläutern.

\subsubsection{Simplexkanäle Anlegen}
\begin{table}[!ht]
 \begin{tabular}{|l|p{2.5cm}|p{2.5cm}|c|c|c|c|} \hline
 Name       & RX Freq. (Ausgabe) & TX Freq. (Eingabe) & TS\footnote{Seteht für \emph{Time Slot} also Zeitschlitz.} & CC\footnote{Steht für \emph{Color Code} also Farbcode.} & TX Kontakt & Empf.gr. \\ \hline
 DMR S0     & $433.4500 MHz$     & $433.4500 MHz$     & 1           & 1        & Rundumruf  & Simplex \\
 DMR S1     & $433.6125 MHz$     & $433.6125 MHz$     & 1           & 1        & Rundumruf  & Simplex \\
 ...        & ...                & ...                & ...         & ...      & ...        & ... \\ \hline
 \end{tabular}
 \caption{Beispieltabelle für die DMR Simplexkanäle.} \label{tab:ch:simplex}
\end{table}

In Tabelle \ref{tab:ch:simplex} sind exemplarisch die Einstellungen der ersten 2 DMR Simplexkanäle aufgeführt. Sie sollten diese natürlich auf alle 8 DMR Simplexkanäle erweitern. Die erste Spalte gibt einfach den Namen des Kanals an. 

Die zweite und dritte Spalte geben die Sende- (TX) und Empfangsfrequenz (RX) des Kanals an. Da es sich hier um Simplexkanäle handelt werden natürlich jeweils die gleichen Frequenzen für RX und TX eingetragen. 

Im \aref{Simplexbetrieb} gibt es keinen Repeater, der den Takt angeben könnte. Daher ist die Wahl des Zeitschlitzes für Simplexkanäle egal. Üblicherweise wird hier einfach der Zeitschlitz 1 ausgewählt.

Der Farbcode (Spalte 5) ist aber nicht egal. Repeater sowie auch Ihr Funkgerät akzeptieren nur dann eine Aussendung, wenn der Farbcode der Aussendung mit der Einstellung für den Kanal übereinstimmt. Bei Simplexkanälen hat man sich daher auf den Farbcode 1 geeinigt. 

Die sechste Spalte gibt den Standardkontakt für diesen Kanal an. Bei Simplexkanälen sollte hier immer der sogenannte. \aref{Rundumruf} (All Call) eingetragen werden. Das bedeutet, dieser \emph{Kontakt} wird immer angerufen, wenn sie diesen Kanal auf dem Funkgerät eingestellt haben und auf die PTT Taste drücken. Eine Ausnahme bildet das Antworten auf einen Ruf. Wenn Sie zum Beispiel einen Gruppenruf zur Sprechgruppe TG99 auf dem Simplexkanal empfangen und innerhalb der kurzen \aref{Hangtime} darauf antworten, werden sie nicht mit dem voreingestellten Rundumruf antworten, sondern mit dem Gruppenruf zur Sprechgruppe TG99. Dieses Verhalten ist sehr erwünscht, da es Ihnen ermöglicht auf auf Direktrufe an Sie mit einem Direktruf zu Antworten.  

Die letzte Spalte gibt die Empfangsgruppe des Kanals an. Damit wird festgelegt welche Sprechgruppen auf diesem Kanal empfangen werden sollen. Wie oben schon erwähnt, wäre hier eigentlich keine Eintragung nötig wenn alle Teilnehmer auf den Simplexkanälen den Rundumruf verwenden würden. Es werden aber durchaus sehr unterschiedliche Sprechgruppen auf den Simplexkanälen verwendet. Für diese Fälle hatten wir ja die Empfangsgruppe \emph{Simplex} zusammengestellt. 

In Ihrer CPS finden sie noch sehr viel mehr Optionen zu den Kanälen. Die Meisten können auf den Standardwerten belassen werden. Am Ende dieses Abschnittes beschreibe ich noch eine Reihe weiterer Optionen. Viele dieser Optionen betreffen Funktionen, die im Amateurfunk aber keine Verwendung finden. 

Die Option \adef{Admit Criterion} definiert unter welchen Umständen das senden auf dem Kanal vom Funkgerät erlaubt wird. Hier stellen Sie bitte \emph{Channel Free} ein. Dies bedeutet, dass sie nur senden dürfen, wenn der Simplexkanal frei ist.  

\subsubsection{Repeaterkanäle Anlegen}
Das Anlegen von Repeater Kanälen ist etwas aufwendiger als das Anlegen von Simplexkanälen, da für jeden Repeater gleich mehrere Kanäle definiert werden. Bevor Sie anfangen können Repeaterkanäle anzulegen, müssen Sie natürlich erst herausfinden welche Repeater sich in Ihrer Nähe befinden. Eine gute Übersicht bietet Ihnen die Seite \url{https://repeatermap.de/}. Sie können dort unter Filter die Anzeige auf DMR Repeater beschränken. Dort finden Sie auch alle wichtigen Information zu den jeweiligen Repeatern. Das heißt deren Eingabe- und Ausgabefrequenzen und Farbcodes. Diese Informationen benötigen Sie unbedingt um Kanäle für diesen Repeater anlegen zu können.

\begin{sidewaystable}[p]
 \centering
 \begin{tabular}{|l|c|c|c|c|c|c|} \hline
  Name             & RX Freq. (Ausgabe) & TX Freq. (Eingabe) & TS\footnote{Seteht für \emph{Time Slot} also Zeitschlitz.} & CC\footnote{Steht für \emph{Color Code} also Farbcode.} & TX Kontakt & Empf.gr. \\ \hline
  DB0LDS TS1       & $439.5625 MHz$ & $431.9625 MHz$ & 1 & 1 & ---         & WW/EU/DL \\
  DB0LDS DL TS1    & $439.5625 MHz$ & $431.9625 MHz$ & 1 & 1 & Deutschland & WW/EU/DL \\
  DB0LDS Sa/Th TS1 & $439.5625 MHz$ & $431.9625 MHz$ & 1 & 1 & Sa/Th       & Sa/Th \\
  DB0LDS TG9 TS2    & $439.5625 MHz$ & $431.9625 MHz$ & 2 & 1 & L9          & Ber/Bra \\
  DB0LDS TG8 TS2    & $439.5625 MHz$ & $431.9625 MHz$ & 2 & 1 & L8          & Ber/Bra \\
  DB0LDS BB TS2    & $439.5625 MHz$ & $431.9625 MHz$ & 2 & 1 & Ber/Bra     & Ber/Bra \\ \hline
 \end{tabular}
 \caption{Beispielkonfiguration der Kanäle für den Repeater DB0LDS in Wildau bei Berlin.} \label{tab:ch:repeater}
\end{sidewaystable}

Ich denke, es ist am einfachsten Ihnen am Beispiel meiner eigenen Kanalliste (Tab. \ref{tab:ch:repeater}) für \textbf{einen} Repeater in meiner Nähe das Anlegen von Repeaterkanälen zu beschreiben. Der Repeater heißt DB0LDS und hat die Eingabe Frequenz $431.9625 MHz$ und die Ausgabe Frequenz $439.5625 MHz$. Des Weiteren erwartet er den Farbcode 1. Dies sind die elementaren Informationen zu diesem Repeater, die Sie von diversen Repeaterlisten und Karten erhalten. Diese Informationen müssen sie natürlich für alle Kanäle die diesen Repeater betreffen, eintragen.

Am Ende des Abschnitts \ref{sec:timeslot} hatte ich erwähnt, dass überregionaler Funkverkehr auf Zeitschlitz 1 und Regionaler auf Zeitschlitz 2 stattfinden. Dies wurde in dieser Konfiguration umgesetzt. 

Der Erste Kanal \emph{DL0LDS TS1} ist ein generischer Kanal für den Zeitschlitz eins. Er besitzt keinen Standardkontakt aber eine Empfangsgruppe für die Sprechgruppen Welt, Europa und Deutschland. Dieser Kanal dient dazu, beliebige (überregionale) Gruppen- und Direktrufe aus der Kontaktliste heraus zu führen. Das heißt, um auf diesem Kanal ein QSO zu starten, kann nicht einfach die PTT Taste gedrückt werden. Denn dazu fehlt dem Kanal der Standardkontakt. Es muss erst ein Kontakt aus der Kontaktliste ausgewählt werden, der angerufen werden soll. 

Der zweite Kanal (\emph{DL0LDS DL TS1}) ist identisch zum Ersten bis auf den Standardkontakt\index{Kanal!Standardkontakt}. Hier ist die Sprechgruppe \emph{Deutschland} (TG262) eingetragen. Das bedeutet, wenn dieser Kanal im Funkgerät ausgewählt ist und die PTT Taste gedrückt wird, wird direkt ein Gruppenruf an diese Sprechgruppe gestartet. Einen extra Kanal für diese Sprechgruppe anzulegen, erlaubt es diesen Gruppenruf zu starten ohne ihn erst in der Kontaktliste auswählen zu müssen. Auch ist es so möglich, diese Sprechgruppe schnell temporär auf diesem Repeater zu abonnieren\footnote{Die Sprechgruppe TG262 (Deutschland) ist für diesen Repeater nicht permanent auf Zeitschlitz 1 abonniert.} indem kurz die PTT Taste gedrückt wird (siehe Abschnitt \ref{sec:talkgroup}).

\begin{merke}
 Auf jedem Kanal kann auf einen eingehenden Ruf innerhalb der sogenannten \aref{Hangtime} geantwortet werden, egal welcher Standardkontakt für diesen Kanal festgelegt wurde. 
\end{merke} 

Ähnlich verhält es sich mit dem dritten Kanal (\emph{DB0LDS Sa/Th TS1}). Hier ist als Standardkontakt die Sprechgruppe \emph{Sachsen/Thüringen} (TG2629) eingestellt um diese schnell und leicht über diesen Repeater erreichen und temporär abonnieren zu können. Bitte beachten Sie, das für diesen Kanal der Zeitschlitz 1 verwendet wird. Der Repeater befindet sich in Brandenburg. Somit sollte die Kommunikation mit Sachsen oder Thüringen im Zeitschlitz für überregionale QSOs geführt werden. Des weiteren ist als Empfangsgruppe die regionale Empfangsgruppe für Sachsen \& Thüringen angegeben. Das bedeutet, dass Gruppenrufe an die überregionalen Sprechgruppen wie \emph{Deutschland} (TG262) auf diesem Kanal nicht empfangen werden, auch wenn er auf Zeitschlitz TS1 liegt.

Kanäle vier, fünf und sechs sind für die lokale (TG9, nur auf diesem Repeater), regionale (TG8, im regionalen Repeaterverbund) und Berlin-Brandenburg-weite Kommunikation (TG2621). All diese Kanäle sind auf Zeitschlitz 2, da es sich um regionale Kommunikation handelt und haben als Empfangsgruppe \emph{Ber/Bra} gesetzt. Das heißt, auf diesen Kanälen werden die Sprechgruppen TG8, TG9 und TG2621 empfangen. Als Standardkontakt wurde die entsprechenden Sprechgruppen gesetzt. Wird auf dem Funkgerät nun der Kanal \emph{DB0LDS TG9 TS2} ausgewählt, so wird beim drücken der PTT Taste ein Gruppenruf an die lokale Sprechgruppe (TG9) von nur diesem Repeater ausgesandt. Wird jedoch der Kanal \emph{DB0LDS BB TS2} ausgewählt, wird beim Drücken der PTT Taste ein Gruppenruf an die Sprachgruppe \emph{Berlin/Brandenburg} (TG2621) gestartet und somit fast überall in Berlin und Brandenburg gehört.

\begin{merke}
 Auf jedem Kanal kann ein beliebiger Ruf (Gruppen, Direkt, Rundum) gestartet werden indem entweder der entsprechende Kontakt in der Kontaktliste ausgewählt wird oder die DMR Nummer eingegeben wird. Dies ist unabhängig vom Standardkontakt des Kanals. Letztendlich dient der Standardkontakt eines Kanals der Bequemlichkeit. So können dedizierte Kanäle für häufig getätigte Rufe definiert werden.
\end{merke}

Das sogenannte \emph{Admit Criterion} sollte für alle Repeaterkanäle auf \emph{Color Code} gesetzt werden. Dies bedeutet, dass Ihr Funkgerät nur dann sendet, wenn der Kanal frei ist und der Farbcode des Repeaters mit dem Farbcode des Kanals übereinstimmt.

\subsubsection{Weitere Kanaloptionen}
Die Maske, mit der Sie Kanäle konfigurieren ist recht umfangreich. Es gibt eine Vielzahl an Optionen die das Verhalten dieses Kanals beeinflussen. Die meisten dieser Optionen werden im Amateurfunk aber nicht verwendet. Dennoch möchte ich diese hier kurz erklären.

Die \aref{Admit Criterion} Option hatte ich zuvor schon erwähnt. Sie liegt fest, unter welchen Umständen das Funkgerät ihnen erlaubt auf dem Kanal zu senden. Meist stehen hier drei Möglichkeiten zu Verfügung. \emph{Always} bedeutet, dass Sie immer senden dürfen. \emph{Channel Free} bedeutet, dass der Kanal frei sein muss, damit Sie senden dürfen. Und \emph{Color Code} bedeutet, dass nicht nur der Kanal frei sein muss, sonder auch der Farbcode des Repeaters stimmen muss. Daher macht es Sinn \emph{Channel Free} für Simplexkanäle und \emph{Color Code} für Repeaterkanäle zu wählen.

Die Option \adef{TOT} oder auch \adef{TX Timeout} legt die maximale Dauer einer Aussendung fest. Das heißt, wenn Sie die PTT Taste länger als diese Zeitspanne drücken, wird das Funkgerät Ihre Aussendung unterbrechen. Dies ist eine Funktion für den kommerziellen Einsatz, die verhindert, dass eine Fehlbedienung das DMR Netz oder auch einen Repeater blockiert. Im Amateurfunk macht dies wenig Sinn. Daher können Sie diese Option auf \emph{unendlich} stellen.

Die Option \adef{Emergency System} legt das Alarm- oder Notrufsystem für diesen Kanal fest. Auch dies ist eine Funktion für den kommerziellen Einsatz und wird im Amateurfunk nicht verwendet.

Die Option \adef{Privacy Group} legt die Verschlüsselung der Aussendungen für diesen Kanal fest. Diese Funktion darf im Amateurfunk gar nicht verwendet werden.

Die Optionen \adef{Emergency Alarm Confirmed}, \adef{Private Call Confirmed} und \adef{Data Call Confirmed}, legen fest, wie das Funkgerät Alarme, Direktrufe und Datenübermittlung durchführt. Sind diese Optionen angewählt, versucht das Funkgerät erst eine Verbindung zum Ziel herzustellen, bevor Sie sprechen dürfen. Das heißt, das Funkgerät sendet zunächst eine Anfrage an das Ziel. Erst wenn diese Anfrage vom Ziel positiv beantwortet wurde, ertönt ein Ton und Sie können sprechen. Wenn diese Optionen nicht angewählt sind, fangen Sie sofort an Sprachdaten an das Ziel zu senden. Ich empfehle Ihnen diese Option nicht anzuwählen.

Die Option \adef{Talkaround} erlaubt es Ihnen auf einem Repeaterkanal Simplexbetrieb zu fahren. Das heißt, Sie werden auf der Repeaterausgabefrequenz senden und empfangen. Dabei umgehen sie natürlich den Repeater selbst. Daher wird dieses Feature im Amateurfunk nicht verwendet.

Wenn die Option \adef{RX Only} angewählt ist, können Sie auf diesen Kanal nicht senden.

Die Option \adef{VOX} bedeutet \altdef{Voice Operated Switch}{VOX} und erlaubt es Ihnen automatisch von Empfang auf Senden umzuschalten sobald sie in das Mikrofon sprechen. Dies lässt sich bei einigen Funkgeräten pro Kanal oder aber auch in den allgemeinen Einstellungen aktivieren.

Die Option \adef{Power} legt die Sendeleistung auf diesen Kanal fest. Wenn Sie sich direkter Nähe des Repeaters befinden, können Sie die Sendeleistung reduzieren um die Batterielaufzeit bei Handfunkgeräten zu erhöhen. 

Diese optionale Kanaleinstellung \aref{Scanlist} definiert welche Liste von Kanälen gescannt werden, wenn der Scan auf diesem Kanal gestartet wird. Es ist nicht zwingend notwendig, dass der Kanal selbst in dieser Scanliste enthalten ist.


\subsection{Zonen Zusammenstellen} \label{sec:zone} \index{Zone}
Wenn Sie nun alle für Sie interessanten Kanäle erstellt haben, werden Sie feststellen, dass die Liste doch schon recht lang und unübersichtlich ist. Alle DMR Funkgeräte organisieren daher die Kanäle in sogenannten Zonen. Diese Zonen sind einfach nur Listen von Kanälen. Wie Sie diese Listen zusammenstellen, ist allein Ihnen überlassen. Sie können die Kanäle nach Region zusammenfassen wie \emph{Zu Hause}, \emph{Arbeit}, \emph{Urlaub} etc.. 

Sie können sie auch nach Sprechgruppen sortieren. So können Sie ein quasi händisches Roaming für eine bestimmte Sprechgruppe realisieren. Wenn Sie dann im Auto unterwegs sind, können Sie im Funkgerät immer jenen Repeater aus der Zone auswählen, den Sie gerade erreichen können. Somit bleiben Sie immer mit dieser Sprechgruppe verbunden. Einige (eher teurere) Funkgeräte unterstützen dies mit einer automatischen Roamingfunktion, bei der der jeweils stärkste Repeater an Ihrem Standort ausgewählt wird.

\begin{hinweis}
 Kanäle, die keiner Zone zugeordnet wurden, können nicht im Funkgerät ausgewählt werden. Es ist aber problemlos möglich, einen Kanal mehreren Zonen zuzuordnen.
\end{hinweis}

\subsection{Scanlisten Zusammenstellen}
Scanlisten sind einfach Listen von Kanälen, die beim Starten der Scanfunktion sequenziell beobachtet werden. Wird ein Signal auf einem der Kanäle empfangen, wird der Scan unterbrochen und es kann dann auf diesen empfangenen Ruf geantwortet werden. Diese Funktion erlaubt es mehrere Kanäle zu beobachten. Zusätzlich können Sie bei vielen Geräten ein oder zwei Prioritätskanäle definieren werden, die während des Scans häufiger \emph{besucht} und somit \emph{intensiver} beobachtet werden.



\section{DMR-Netze} \label{sec:netze}
Sie kennen nun alle wichtigen Konzepte des DMR-Betriebs und auch einige der technischen Details dazu, wie das Erstellen von Codeplugs. Diese Konzepte gelten jedoch nur uneingeschränkt im sogenannten Brandmeisternetz. Dies ist jenes Netz im Hintergrund, dass ihre Direkt- oder Gruppenrufe vermittelt und Repeater miteinander verbindet. In Deutschland ist dies das dominierende Netz. Auch Weltweit sind die meisten DMR Repeater (c.a., 1500) im Brandmeisternetz miteinander verbunden. Es gibt aber auch andere DMR Netze. Zum Einen DMR-MARC (c.a., 500 Repeater) und zum Anderen DMR+ (c.a. 150 Repeater). Welches Netz wo häufiger verwendet wird, hängt stark vom Land ab. So sind in Frankreich, Spanien, den BeNeLux Staaten, Polen, Tschechien und der Slowakei fasst ausschließlich Brandmeister Repeater im Betrieb. Während in Dänemark DMR+ deutlich mehr Repeater vernetzt. In Großbritannien, den USA und Österreich sind DMR-MARC Repeater nicht selten. All diese Netze unterscheiden sich aber nicht technisch voneinander. Das heißt, die Ihnen zugewiesene DMR-ID ist in allen Netzen gültig und sie können jedes Tier-II DMR Funkgerät in allen Netzen verwenden. 

Lediglich die Konzepte der einzelnen Netze, vor allem wie Gruppenrufe realisiert werden, ist von Netz zu Netz verschieden. Das heißt, Sie müssen die Kanäle zu einem DMR+ Repeater leicht anders konfigurieren als Kanäle zu einem Brandmeister Repeater. 

\subsection{Reflektoren} \label{sec:reflector} \index{Reflektor}
Im DMR+ Netz spielen sogenannte Reflektoren eine zentrale Rolle. Sie entsprechen in etwa den Sprechgruppen, wie sie im Brandmeisternetz verwendet werden. 

Der wesentliche Unterschied zu Sprechgruppen im Brandmeister Netz ist, dass diese Reflektoren nicht einfach per Gruppenruf angerufen werden können, sondern zunächst per Direktruf an einem Repeater temporär abonniert werden müssen. Danach verhalten sich alle Repeater, die diesen Reflektor abonniert haben, wie eine Gruppe zusammen geschalteter FM Repeater. Das heißt, ein Gruppenruf zur lokalen Sprechgruppe TG9 wird dann nicht nur lokal ausgesandt, sondern auch über alle Repeater die diesen Reflektor abonniert haben.

Dies hat den Vorteil, dass die Konfiguration des Funkgerätes viel einfacher ist: Es müssen lediglich zwei Kanäle für jeden Repeater angelegt werden. Je einen für jeden Zeitschlitz und jeweils mit dem Standardkontakt zur TG9. Um einen Reflektor am aktuellen Repeater zu abonnieren, wird einfach ein Direktruf zu dem Reflektor aus der Kontaktliste heraus gestartet. Dieses Konzept ist auch näher an den \emph{alten} Konzepten aus dem FM Repeaterbetrieb mit Echolink. Jedoch gehen dadurch modernere Fähigkeiten des Netzes wie Roaming verloren. Dieses Konzept hat aber auch den Nachteil, dass die Repeatertransparenz verloren geht. Anstatt einfach einen Gruppenruf zu der gewünschten Sprechgruppe zu starten, muss zunächst der lokale Repeater \emph{konfiguriert} werden. Erst danach erfolgt alle Kommunikation über die lokale Sprechgruppe TG9, auch wenn diese Kommunikation alles andere als lokal ist. 

\appendix
\printindex

\end{document}