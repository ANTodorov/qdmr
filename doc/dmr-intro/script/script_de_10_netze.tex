\section{DMR-Netze} \label{sec:netze}
Sie kennen nun alle wichtigen Konzepte des DMR-Betriebs und auch einige der technischen Details dazu, wie das Erstellen von Codeplugs. Diese Konzepte gelten jedoch nur uneingeschränkt im sogenannten Brandmeisternetz. Dies ist jenes Netz im Hintergrund, dass ihre Direkt- oder Gruppenrufe vermittelt und Repeater miteinander verbindet. In Deutschland ist dies das dominierende Netz. Auch Weltweit sind die meisten DMR Repeater (c.a., 1500) im Brandmeisternetz miteinander verbunden. Es gibt aber auch andere DMR Netze. Zum Einen DMR-MARC (c.a., 500 Repeater) und zum Anderen DMR+ (c.a. 150 Repeater). Welches Netz wo häufiger verwendet wird, hängt stark vom Land ab. So sind in Frankreich, Spanien, den BeNeLux Staaten, Polen, Tschechien und der Slowakei fasst ausschließlich Brandmeister Repeater im Betrieb. Während in Dänemark DMR+ deutlich mehr Repeater vernetzt. In Großbritannien, den USA und Österreich sind DMR-MARC Repeater nicht selten. All diese Netze unterscheiden sich aber nicht technisch voneinander. Das heißt, die Ihnen zugewiesene DMR-ID ist in allen Netzen gültig und sie können jedes Tier-II DMR Funkgerät in allen Netzen verwenden. 

Lediglich die Konzepte der einzelnen Netze, vor allem wie Gruppenrufe realisiert werden, ist von Netz zu Netz verschieden. Das heißt, Sie müssen die Kanäle zu einem DMR+ Repeater leicht anders konfigurieren als Kanäle zu einem Brandmeister Repeater. 

\subsection{Reflektoren} \label{sec:reflector} \index{Reflektor}
Im DMR+ Netz spielen sogenannte Reflektoren eine zentrale Rolle. Sie entsprechen in etwa den Sprechgruppen, wie sie im Brandmeisternetz verwendet werden. 

Der wesentliche Unterschied zu Sprechgruppen im Brandmeister Netz ist, dass diese Reflektoren nicht einfach per Gruppenruf angerufen werden können, sondern zunächst per Direktruf an einem Repeater temporär abonniert werden müssen. Danach verhalten sich alle Repeater, die diesen Reflektor abonniert haben, wie eine Gruppe zusammen geschalteter FM Repeater. Das heißt, ein Gruppenruf zur lokalen Sprechgruppe TG9 wird dann nicht nur lokal ausgesandt, sondern auch über alle Repeater die diesen Reflektor abonniert haben.

Dies hat den Vorteil, dass die Konfiguration des Funkgerätes viel einfacher ist: Es müssen lediglich zwei Kanäle für jeden Repeater angelegt werden. Je einen für jeden Zeitschlitz und jeweils mit dem Standardkontakt zur TG9. Um einen Reflektor am aktuellen Repeater zu abonnieren, wird einfach ein Direktruf zu dem Reflektor aus der Kontaktliste heraus gestartet. Dieses Konzept ist auch näher an den \emph{alten} Konzepten aus dem FM Repeaterbetrieb mit Echolink. Jedoch gehen dadurch modernere Fähigkeiten des Netzes wie Roaming verloren. Dieses Konzept hat aber auch den Nachteil, dass die Repeatertransparenz verloren geht. Anstatt einfach einen Gruppenruf zu der gewünschten Sprechgruppe zu starten, muss zunächst der lokale Repeater \emph{konfiguriert} werden. Erst danach erfolgt alle Kommunikation über die lokale Sprechgruppe TG9, auch wenn diese Kommunikation alles andere als lokal ist. 