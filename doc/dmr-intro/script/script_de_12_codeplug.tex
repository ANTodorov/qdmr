\section{Codeplug Programmierung} \label{sec:codeplug}
Nachdem Sie sich mit den Konzepten und dem technischen Hintergrund von DMR auseinandergesetzt haben, geht es nun an die Konfiguration Ihres Funkgerätes. Dies geschieht üblicherweise nicht über das Bedienfeld des Funkgerätes, sonder mit Hilfe einer separaten Software, der sogenannten \adef{CPS} oder \emph{codeplug programming software}. 

Doch bevor Sie loslegen können benötigen Sie wie alle DMR Teilnehmer eine eindeutige Nummer, die DMR ID.
\begin{hinweis}
 Ihre persönliche und eindeutige DMR ID erhalten Sie unter \url{https://register.ham-digital.org/}. Da Sie nachweisen müssen, dass Sie lizenzierter Funkamateur sind, müssen Sie bei der Anmeldung ihre eingescannte \emph{Zulassung zum Amateurfunkdienst} hochladen.
\end{hinweis}
Ihre DMR ID erhalten Sie in der Regel innerhalb von 24 Stunden per Mail. Sobald Sie eine DMR ID erhalten haben kann es los gehen.

Da dieses Script für Einsteiger gedacht ist, ist es wahrscheinlich, dass Sie kein top-shelf Motorola Gerät sondern eher ein günstiges Gerät der einschlägig bekannten chinesischen Hersteller besitzen. 

\begin{achtung}
 Falls Sie noch kein DMR fähiges Funkgerät besitzen und mit dem Gedanken spielen eines zu kaufen, achten Sie unbedingt darauf, dass es DMR \textbf{Tier I \& II}\footnote{Wie so häufig ist DMR nicht ein Standard sondern eine ganze Familie von aufeinander aufbauenden Standards. DMR Tier I beschreibt im wesentlichen den DMR Simplexbetrieb und Tier II dann den Repeaterbetrieb mit zwei Zeitschlitzen. Sie benötigen also unbedingt Tier II für den Repeaterbetrieb.} unterstützt. Ignorieren Sie etwaiges Marketing-Bla-Bla der Hersteller und schauen Sie in den technischen Details nach, ob dort DMR \textbf{Tier I \& II} erwähnt wird. Falls nicht oder nicht eindeutig, lassen Sie die Finger von diesem Gerät! Dies gilt vor allem für das Baofeng MD-5R aber nicht für das Baofeng/Radioddity RD-5R\footnote{Manchmal sind es die kleinen Unterschiede die entscheidend sind.}. 
\end{achtung}

Der Hersteller Ihres Gerätes wird auf seiner Webseite die Software die Sie zur Konfiguration benötigen, zum Download bereitstellen. Diese Software wird \emph{CPS} oder \emph{codeplug programming software} genannt. Gegebenenfalls finden Sie dort auch Firmwareupdates für Ihr Gerät. Viele Hersteller bieten für jedes einzelne Modell eine separate CPS an oder gar für jede Variation eines Modells. Achten Sie also genau darauf welche CPS Sie herunterladen. Die Konfiguration dieser Geräte unterscheidet sich von Gerät zu Gerät und mehr noch von Hersteller zu Hersteller. Jedoch sind die wesentlichen Einstellungen für Geräte dieser Klasse sehr ähnlich.

Wenn Sie die CPS zum ersten mal starten, werden Sie wahrscheinlich zwei Dinge feststellen. Erstens, das Bedienkonzept dieser Software ist aus dem letzten Jahrtausend (Windows 3.11) und Zweitens, es gibt eine Unmenge an obskuren Optionen deren Funktion nicht ersichtlich ist und die größtenteils nicht Dokumentiert sind. Wenn Sie des Englischen nicht mächtig sind, werden Sie auch eine deutsche Übersetzung des Programms vermissen. Aber keine sorge, die englische Übersetzung ist meist auch so schlecht, dass es keinen Unterschied macht ob sie Englisch lesen können oder nicht.

Die Konfiguration Ihres Funkgerätes erfolgt in 5-6 Schritten:
\begin{enumerate}
 \item Allgemeine Einstellungen,
 \item Kontakte anlegen,
 \item Empfangsgruppen festlegen,
 \item alle Kanäle anlegen,
 \item Kanäle in Zonen einteilen und
 \item optional Scanlisten anlegen.
\end{enumerate}

In den folgenden Abschnitten möchte ich die einzelnen Konfigurationsschritte im Detail beschreiben.

\subsection{Allgemeine Konfiguration} \label{sec:cp:basic}
Die wichtigsten allgemeinen Einstellungen die Sie vornehmen müssen, ist das setzen der DMR ID und ihres Rufzeichens. Diese Optionen finden Sie meist unter der Rubrik (linke Seite) \emph{Radio Settings} oder \emph{General Settings}\footnote{Die exakten Namen der Rubriken und Felder kann sich von Hersteller zu Hersteller unterscheiden. Üblicherweise sind sie aber den hier erwähnten Namen sehr ähnlich.}. Ihre DMR ID tragen Sie dann in das Feld \emph{Radio ID} ein. Es ist durchaus möglich, dass Ihr Funkgerät mehrere DMR IDs unterstützt. Dieses Feature wird aber üblicherweise nicht verwendet. Im Gegenteil: Es stehen nur eine begrenzte Anzahl von DMR IDs sehr vielen Funkamateuren gegenüber. Beantragen sie deshalb niemals eine DMR ID für jedes Funkgerät oder jeden Accesspoint. Eine \textbf{einzige} persönliche DMR Nummer reicht völlig! 

Ihr Rufzeichen tragen Sie bitte in das Feld \emph{Radio Name} ebenfalls in der Rubrik \emph{Radio Settings} ein.  

\subsection{Kontakte Anlegen} \label{sec:cp:contact}
Nachdem Sie die grundlegenden Einstellungen vorgenommen haben, können Sie Ihre Kontaktliste zusammenstellen. Diese sollte alle Sprechgruppen enthalten die Sie interessieren könnten, ihre persönlichen Kontakte wie OMs aus dem OV und einige Servicenummern wie Echo, die SMS Dienste und den All Call. Eine Beispiel für Deutschland ist in Tabelle \ref{tab:contacts} angegeben.

\begin{table}[!ht]
 \centering
 \begin{tabular}{|l|c|c||l|c|c|}\hline
  Name        & Typ        & Nummer & Name & Typ & Nummer \\ \hline
  Lokal       & Gruppenruf & 9        & Ham/SlHo    & Gruppenruf & 2622 \\
  Regional    & Gruppenruf & 8        & NiSa/Bre    & Gruppenruf & 2623 \\
  TG99        & Gruppenruf & 99       & NRW         & Gruppenruf & 2624 \\
  Rundumruf   & All Call   & 16777215 & RhPf/Saar   & Gruppenruf & 2625 \\
  Weltweit    & Gruppenruf & 91       & Hessen      & Gruppenruf & 2626 \\
  Europa      & Gruppenruf & 92       & BaWü        & Gruppenruf & 2627 \\
  D-A-CH      & Gruppenruf & 920      & Bay         & Gruppenruf & 2628 \\
  Deutschland & Gruppenruf & 262      & Sa/Th       & Gruppenruf & 2629 \\
  Österreich  & Gruppenruf & 232      & Echo Test   & Direktruf  & 262997 \\
  Schweiz     & Gruppenruf & 228      & SMS Serv.   & Direktruf  & 262993 \\
  EMCOM\footnote{Ausschließlich für Notfunk.} EU    & Gruppenruf & 9112 & 
  DAPNET      & Direktruf  & 262994 \\
  EMCOM WW    & Gruppenruf & 9911     & APRS GW     & Direktruf  & 262999 \\
  MeVo/SaAn   & Gruppenruf & 2620     & DM3MAT      & Direktruf  & 2621370 \\
  Ber/Bra     & Gruppenruf & 2621     & ...         & ...        & ... \\ \hline
 \end{tabular}
 \caption{Beispielkontakte für Deutschland.} \label{tab:contacts}
\end{table}

Natürlich gibt es noch viele weitere Sprechgruppen auch zu spezifischen Themen, die nicht unbedingt regional beschränkt sind. Eine recht vollständige Liste finden Sie unter \url{https://www.pistar.uk/dmr_bm_talkgroups.php}.


\subsection{Empfangsgruppen Zusammenstellen} \label{sec:cp:grouplist}
Im nächsten Schritt stellen Sie sogenannte \adef{Empfangsgruppen} zusammen. Dies sind Listen von Gruppenrufen, die Sie auf bestimmten Kanälen empfangen wollen. Wie schon bei der Einführung in Abschnitt \ref{sec:ursprung} erwähnt, weiß das DMR Netz nicht, für welche Sprechgruppen Sie sich interessieren. Dies kann nur Ihr Funkgerät wissen. Mit den Empfangsgruppen definieren Sie genau das. Sie werden mindestens drei Empfangsgruppen benötigen. Eine für den Simplexbetrieb, eine für die überregionale Kommunikation und je eine für regionale Kommunikation in all jenen Regionen, in denen Sie unterwegs sind.

Die Simplex Empfangsgruppe ist eigentlich nicht notwendig, da Simplexrufe eigentlich immer den sog. \aref{All Call} (Rundumruf) verwenden sollten. Häufig wird aber auch die Sprechgruppe TG99, TG9 oder auch TG8 verwendet. Daher ist es ratsam eine Empfangsgruppe mit diesen Gruppenrufen anzulegen. 

Für die überregionale Kommunikation sollte eine Empfangsgruppe erstellt werden, die die Sprechgruppen für weltweite, innereuropäische und deutschlandweite Kommunikation enthalten. Dieser Gruppe können Sie dann noch die Sprechgruppe \emph{EMCOM EU} für europäischen Notfunk hinzufügen, damit Sie ggf. Notrufe hören und darauf reagieren können. 

Zuletzt sollte die Sprechgruppen für lokale/regionale Kommunikation angelegt werde. Diese sollte jeweils die Sprechgruppen TG8 und TG9 sowie die Sprechgruppe der jeweiligen Region enthalten. Für mich, der in der Berlin/Brandenburg Region lebt, aber häufig auch in Sachsen unterwegs ist, habe ich insgesamt 4 Empfangsgruppen zusammengestellt (siehe Tab. \ref{tab:grouplist}).
  
\begin{table}
 \centering
 \begin{tabular}{|l|l|} \hline
 Name     & Gruppenrufe \\ \hline
 Simplex  & Lokal, Regional, TG99 \\
 WW/EU/DL & Weltweit, Europa, D-A-CH, Deutschland, EMCOM EU \\
 Ber/Bra  & Lokal, Regional, Ber/Bra \\
 Sa/Th    & Lokal, Regional, Sa/Th \\ \hline
 \end{tabular}
 \caption{Ein paar Beispielempfangsgruppen. Die ersten beiden sind recht universell für Deutschland, die letzten Beiden sind für die Regionen Berlin/Brandenburg und Sachsen/Thüringen wichtig.} \label{tab:grouplist}
\end{table}


\subsection{Kanäle Anlegen} \label{sec:cp:channel}
Bevor es los geht, sollte ich erwähnen, dass die meisten DMR Funkgeräte auch analoges FM unterstützen. Das heißt, Sie können mit ihrem DMR Funkgerät auch normalen analogen FM Simplex und Repeaterbetrieb durchführen. In diesem Abschnitt beschreibe ich aber nur die Konfiguration von DMR Kanälen (meist \emph{Digital Channel} genannt), die Konfiguration von sogenannten \emph{analogen} Kanälen wird hier nicht beschrieben. Um einen DMR Kanal anzulegen, müssen Sie im Feld \emph{Channel Type} den Wert \emph{digital} auswählen, für einen FM Kanal dann \emph{analog}.

Wenn Sie schon Erfahrung mit dem \emph{klassischen} FM-Relaisbetrieb haben, wird Ihnen das Anlegen der Kanäle recht seltsam vorkommen. Im analogen FM-Relaisbetrieb haben Sie für jeden Repeater und Simplex-Kanal genau einen Kanal im Funkgerät konfiguriert. Für den DMR Betrieb werden Sie für jeden Repeater mindestens zwei (für Zeitschlitz 1 \& 2), meist aber deutlich mehr Kanäle programmieren. Lange Rede kurzer Sinn. Lassen Sie mich das an konkreten Beispielen erläutern.

\subsubsection{Simplexkanäle Anlegen}
\begin{table}[!ht]
 \begin{tabular}{|l|p{2.5cm}|p{2.5cm}|c|c|c|c|} \hline
 Name       & RX Freq. (Ausgabe) & TX Freq. (Eingabe) & TS\footnote{Seteht für \emph{Time Slot} also Zeitschlitz.} & CC\footnote{Steht für \emph{Color Code} also Farbcode.} & TX Kontakt & Empf.gr. \\ \hline
 DMR S0     & $433.4500 MHz$     & $433.4500 MHz$     & 1           & 1        & Rundumruf  & Simplex \\
 DMR S1     & $433.6125 MHz$     & $433.6125 MHz$     & 1           & 1        & Rundumruf  & Simplex \\
 ...        & ...                & ...                & ...         & ...      & ...        & ... \\ \hline
 \end{tabular}
 \caption{Beispieltabelle für die DMR Simplexkanäle.} \label{tab:ch:simplex}
\end{table}

In Tabelle \ref{tab:ch:simplex} sind exemplarisch die Einstellungen der ersten 2 DMR Simplexkanäle aufgeführt. Sie sollten diese natürlich auf alle 8 DMR Simplexkanäle erweitern. Die erste Spalte gibt einfach den Namen des Kanals an. 

Die zweite und dritte Spalte geben die Sende- (TX) und Empfangsfrequenz (RX) des Kanals an. Da es sich hier um Simplexkanäle handelt werden natürlich jeweils die gleichen Frequenzen für RX und TX eingetragen. 

Im \aref{Simplexbetrieb} gibt es keinen Repeater, der den Takt angeben könnte. Daher ist die Wahl des Zeitschlitzes für Simplexkanäle egal. Üblicherweise wird hier einfach der Zeitschlitz 1 ausgewählt.

Der Farbcode (Spalte 5) ist aber nicht egal. Repeater sowie auch Ihr Funkgerät akzeptieren nur dann eine Aussendung, wenn der Farbcode der Aussendung mit der Einstellung für den Kanal übereinstimmt. Bei Simplexkanälen hat man sich daher auf den Farbcode 1 geeinigt. 

Die sechste Spalte gibt den Standardkontakt für diesen Kanal an. Bei Simplexkanälen sollte hier immer der sogenannte. \aref{Rundumruf} (All Call) eingetragen werden. Das bedeutet, dieser \emph{Kontakt} wird immer angerufen, wenn sie diesen Kanal auf dem Funkgerät eingestellt haben und auf die PTT Taste drücken. Eine Ausnahme bildet das Antworten auf einen Ruf. Wenn Sie zum Beispiel einen Gruppenruf zur Sprechgruppe TG99 auf dem Simplexkanal empfangen und innerhalb der kurzen \aref{Hangtime} darauf antworten, werden sie nicht mit dem voreingestellten Rundumruf antworten, sondern mit dem Gruppenruf zur Sprechgruppe TG99. Dieses Verhalten ist sehr erwünscht, da es Ihnen ermöglicht auf auf Direktrufe an Sie mit einem Direktruf zu Antworten.  

Die letzte Spalte gibt die Empfangsgruppe des Kanals an. Damit wird festgelegt welche Sprechgruppen auf diesem Kanal empfangen werden sollen. Wie oben schon erwähnt, wäre hier eigentlich keine Eintragung nötig wenn alle Teilnehmer auf den Simplexkanälen den Rundumruf verwenden würden. Es werden aber durchaus sehr unterschiedliche Sprechgruppen auf den Simplexkanälen verwendet. Für diese Fälle hatten wir ja die Empfangsgruppe \emph{Simplex} zusammengestellt. 

In Ihrer CPS finden sie noch sehr viel mehr Optionen zu den Kanälen. Die Meisten können auf den Standardwerten belassen werden. Am Ende dieses Abschnittes beschreibe ich noch eine Reihe weiterer Optionen. Viele dieser Optionen betreffen Funktionen, die im Amateurfunk aber keine Verwendung finden. 

Die Option \adef{Admit Criterion} definiert unter welchen Umständen das senden auf dem Kanal vom Funkgerät erlaubt wird. Hier stellen Sie bitte \emph{Channel Free} ein. Dies bedeutet, dass sie nur senden dürfen, wenn der Simplexkanal frei ist.  

\subsubsection{Repeaterkanäle Anlegen}
Das Anlegen von Repeater Kanälen ist etwas aufwendiger als das Anlegen von Simplexkanälen, da für jeden Repeater gleich mehrere Kanäle definiert werden. Bevor Sie anfangen können Repeaterkanäle anzulegen, müssen Sie natürlich erst herausfinden welche Repeater sich in Ihrer Nähe befinden. Eine gute Übersicht bietet Ihnen die Seite \url{https://repeatermap.de/}. Sie können dort unter Filter die Anzeige auf DMR Repeater beschränken. Dort finden Sie auch alle wichtigen Information zu den jeweiligen Repeatern. Das heißt deren Eingabe- und Ausgabefrequenzen und Farbcodes. Diese Informationen benötigen Sie unbedingt um Kanäle für diesen Repeater anlegen zu können.

\begin{sidewaystable}[p]
 \centering
 \begin{tabular}{|l|c|c|c|c|c|c|} \hline
  Name             & RX Freq. (Ausgabe) & TX Freq. (Eingabe) & TS\footnote{Seteht für \emph{Time Slot} also Zeitschlitz.} & CC\footnote{Steht für \emph{Color Code} also Farbcode.} & TX Kontakt & Empf.gr. \\ \hline
  DB0LDS TS1       & $439.5625 MHz$ & $431.9625 MHz$ & 1 & 1 & ---         & WW/EU/DL \\
  DB0LDS DL TS1    & $439.5625 MHz$ & $431.9625 MHz$ & 1 & 1 & Deutschland & WW/EU/DL \\
  DB0LDS Sa/Th TS1 & $439.5625 MHz$ & $431.9625 MHz$ & 1 & 1 & Sa/Th       & Sa/Th \\
  DB0LDS TG9 TS2    & $439.5625 MHz$ & $431.9625 MHz$ & 2 & 1 & L9          & Ber/Bra \\
  DB0LDS TG8 TS2    & $439.5625 MHz$ & $431.9625 MHz$ & 2 & 1 & L8          & Ber/Bra \\
  DB0LDS BB TS2    & $439.5625 MHz$ & $431.9625 MHz$ & 2 & 1 & Ber/Bra     & Ber/Bra \\ \hline
 \end{tabular}
 \caption{Beispielkonfiguration der Kanäle für den Repeater DB0LDS in Wildau bei Berlin.} \label{tab:ch:repeater}
\end{sidewaystable}

Ich denke, es ist am einfachsten Ihnen am Beispiel meiner eigenen Kanalliste (Tab. \ref{tab:ch:repeater}) für \textbf{einen} Repeater in meiner Nähe das Anlegen von Repeaterkanälen zu beschreiben. Der Repeater heißt DB0LDS und hat die Eingabe Frequenz $431.9625 MHz$ und die Ausgabe Frequenz $439.5625 MHz$. Des Weiteren erwartet er den Farbcode 1. Dies sind die elementaren Informationen zu diesem Repeater, die Sie von diversen Repeaterlisten und Karten erhalten. Diese Informationen müssen sie natürlich für alle Kanäle die diesen Repeater betreffen, eintragen.

Am Ende des Abschnitts \ref{sec:timeslot} hatte ich erwähnt, dass überregionaler Funkverkehr auf Zeitschlitz 1 und Regionaler auf Zeitschlitz 2 stattfinden. Dies wurde in dieser Konfiguration umgesetzt. 

Der Erste Kanal \emph{DL0LDS TS1} ist ein generischer Kanal für den Zeitschlitz eins. Er besitzt keinen Standardkontakt aber eine Empfangsgruppe für die Sprechgruppen Welt, Europa und Deutschland. Dieser Kanal dient dazu, beliebige (überregionale) Gruppen- und Direktrufe aus der Kontaktliste heraus zu führen. Das heißt, um auf diesem Kanal ein QSO zu starten, kann nicht einfach die PTT Taste gedrückt werden. Denn dazu fehlt dem Kanal der Standardkontakt. Es muss erst ein Kontakt aus der Kontaktliste ausgewählt werden, der angerufen werden soll. 

Der zweite Kanal (\emph{DL0LDS DL TS1}) ist identisch zum Ersten bis auf den Standardkontakt\index{Kanal!Standardkontakt}. Hier ist die Sprechgruppe \emph{Deutschland} (TG262) eingetragen. Das bedeutet, wenn dieser Kanal im Funkgerät ausgewählt ist und die PTT Taste gedrückt wird, wird direkt ein Gruppenruf an diese Sprechgruppe gestartet. Einen extra Kanal für diese Sprechgruppe anzulegen, erlaubt es diesen Gruppenruf zu starten ohne ihn erst in der Kontaktliste auswählen zu müssen. Auch ist es so möglich, diese Sprechgruppe schnell temporär auf diesem Repeater zu abonnieren\footnote{Die Sprechgruppe TG262 (Deutschland) ist für diesen Repeater nicht permanent auf Zeitschlitz 1 abonniert.} indem kurz die PTT Taste gedrückt wird (siehe Abschnitt \ref{sec:talkgroup}).

\begin{merke}
 Auf jedem Kanal kann auf einen eingehenden Ruf innerhalb der sogenannten \aref{Hangtime} geantwortet werden, egal welcher Standardkontakt für diesen Kanal festgelegt wurde. 
\end{merke} 

Ähnlich verhält es sich mit dem dritten Kanal (\emph{DB0LDS Sa/Th TS1}). Hier ist als Standardkontakt die Sprechgruppe \emph{Sachsen/Thüringen} (TG2629) eingestellt um diese schnell und leicht über diesen Repeater erreichen und temporär abonnieren zu können. Bitte beachten Sie, das für diesen Kanal der Zeitschlitz 1 verwendet wird. Der Repeater befindet sich in Brandenburg. Somit sollte die Kommunikation mit Sachsen oder Thüringen im Zeitschlitz für überregionale QSOs geführt werden. Des weiteren ist als Empfangsgruppe die regionale Empfangsgruppe für Sachsen \& Thüringen angegeben. Das bedeutet, dass Gruppenrufe an die überregionalen Sprechgruppen wie \emph{Deutschland} (TG262) auf diesem Kanal nicht empfangen werden, auch wenn er auf Zeitschlitz TS1 liegt.

Kanäle vier, fünf und sechs sind für die lokale (TG9, nur auf diesem Repeater), regionale (TG8, im regionalen Repeaterverbund) und Berlin-Brandenburg-weite Kommunikation (TG2621). All diese Kanäle sind auf Zeitschlitz 2, da es sich um regionale Kommunikation handelt und haben als Empfangsgruppe \emph{Ber/Bra} gesetzt. Das heißt, auf diesen Kanälen werden die Sprechgruppen TG8, TG9 und TG2621 empfangen. Als Standardkontakt wurde die entsprechenden Sprechgruppen gesetzt. Wird auf dem Funkgerät nun der Kanal \emph{DB0LDS TG9 TS2} ausgewählt, so wird beim drücken der PTT Taste ein Gruppenruf an die lokale Sprechgruppe (TG9) von nur diesem Repeater ausgesandt. Wird jedoch der Kanal \emph{DB0LDS BB TS2} ausgewählt, wird beim Drücken der PTT Taste ein Gruppenruf an die Sprachgruppe \emph{Berlin/Brandenburg} (TG2621) gestartet und somit fast überall in Berlin und Brandenburg gehört.

\begin{merke}
 Auf jedem Kanal kann ein beliebiger Ruf (Gruppen, Direkt, Rundum) gestartet werden indem entweder der entsprechende Kontakt in der Kontaktliste ausgewählt wird oder die DMR Nummer eingegeben wird. Dies ist unabhängig vom Standardkontakt des Kanals. Letztendlich dient der Standardkontakt eines Kanals der Bequemlichkeit. So können dedizierte Kanäle für häufig getätigte Rufe definiert werden.
\end{merke}

Das sogenannte \emph{Admit Criterion} sollte für alle Repeaterkanäle auf \emph{Color Code} gesetzt werden. Dies bedeutet, dass Ihr Funkgerät nur dann sendet, wenn der Kanal frei ist und der Farbcode des Repeaters mit dem Farbcode des Kanals übereinstimmt.

\subsubsection{Weitere Kanaloptionen}
Die Maske, mit der Sie Kanäle konfigurieren ist recht umfangreich. Es gibt eine Vielzahl an Optionen die das Verhalten dieses Kanals beeinflussen. Die meisten dieser Optionen werden im Amateurfunk aber nicht verwendet. Dennoch möchte ich diese hier kurz erklären.

Die \aref{Admit Criterion} Option hatte ich zuvor schon erwähnt. Sie liegt fest, unter welchen Umständen das Funkgerät ihnen erlaubt auf dem Kanal zu senden. Meist stehen hier drei Möglichkeiten zu Verfügung. \emph{Always} bedeutet, dass Sie immer senden dürfen. \emph{Channel Free} bedeutet, dass der Kanal frei sein muss, damit Sie senden dürfen. Und \emph{Color Code} bedeutet, dass nicht nur der Kanal frei sein muss, sonder auch der Farbcode des Repeaters stimmen muss. Daher macht es Sinn \emph{Channel Free} für Simplexkanäle und \emph{Color Code} für Repeaterkanäle zu wählen.

Die Option \adef{TOT} oder auch \adef{TX Timeout} legt die maximale Dauer einer Aussendung fest. Das heißt, wenn Sie die PTT Taste länger als diese Zeitspanne drücken, wird das Funkgerät Ihre Aussendung unterbrechen. Dies ist eine Funktion für den kommerziellen Einsatz, die verhindert, dass eine Fehlbedienung das DMR Netz oder auch einen Repeater blockiert. Im Amateurfunk macht dies wenig Sinn. Daher können Sie diese Option auf \emph{unendlich} stellen.

Die Option \adef{Emergency System} legt das Alarm- oder Notrufsystem für diesen Kanal fest. Auch dies ist eine Funktion für den kommerziellen Einsatz und wird im Amateurfunk nicht verwendet.

Die Option \adef{Privacy Group} legt die Verschlüsselung der Aussendungen für diesen Kanal fest. Diese Funktion darf im Amateurfunk gar nicht verwendet werden.

Die Optionen \adef{Emergency Alarm Confirmed}, \adef{Private Call Confirmed} und \adef{Data Call Confirmed}, legen fest, wie das Funkgerät Alarme, Direktrufe und Datenübermittlung durchführt. Sind diese Optionen angewählt, versucht das Funkgerät erst eine Verbindung zum Ziel herzustellen, bevor Sie sprechen dürfen. Das heißt, das Funkgerät sendet zunächst eine Anfrage an das Ziel. Erst wenn diese Anfrage vom Ziel positiv beantwortet wurde, ertönt ein Ton und Sie können sprechen. Wenn diese Optionen nicht angewählt sind, fangen Sie sofort an Sprachdaten an das Ziel zu senden. Ich empfehle Ihnen diese Option nicht anzuwählen.

Die Option \adef{Talkaround} erlaubt es Ihnen auf einem Repeaterkanal Simplexbetrieb zu fahren. Das heißt, Sie werden auf der Repeaterausgabefrequenz senden und empfangen. Dabei umgehen sie natürlich den Repeater selbst. Daher wird dieses Feature im Amateurfunk nicht verwendet.

Wenn die Option \adef{RX Only} angewählt ist, können Sie auf diesen Kanal nicht senden.

Die Option \adef{VOX} bedeutet \altdef{Voice Operated Switch}{VOX} und erlaubt es Ihnen automatisch von Empfang auf Senden umzuschalten sobald sie in das Mikrofon sprechen. Dies lässt sich bei einigen Funkgeräten pro Kanal oder aber auch in den allgemeinen Einstellungen aktivieren.

Die Option \adef{Power} legt die Sendeleistung auf diesen Kanal fest. Wenn Sie sich direkter Nähe des Repeaters befinden, können Sie die Sendeleistung reduzieren um die Batterielaufzeit bei Handfunkgeräten zu erhöhen. 

Diese optionale Kanaleinstellung \aref{Scanlist} definiert welche Liste von Kanälen gescannt werden, wenn der Scan auf diesem Kanal gestartet wird. Es ist nicht zwingend notwendig, dass der Kanal selbst in dieser Scanliste enthalten ist.


\subsection{Zonen Zusammenstellen} \label{sec:zone} \index{Zone}
Wenn Sie nun alle für Sie interessanten Kanäle erstellt haben, werden Sie feststellen, dass die Liste doch schon recht lang und unübersichtlich ist. Alle DMR Funkgeräte organisieren daher die Kanäle in sogenannten Zonen. Diese Zonen sind einfach nur Listen von Kanälen. Wie Sie diese Listen zusammenstellen, ist allein Ihnen überlassen. Sie können die Kanäle nach Region zusammenfassen wie \emph{Zu Hause}, \emph{Arbeit}, \emph{Urlaub} etc.. 

Sie können sie auch nach Sprechgruppen sortieren. So können Sie ein quasi händisches Roaming für eine bestimmte Sprechgruppe realisieren. Wenn Sie dann im Auto unterwegs sind, können Sie im Funkgerät immer jenen Repeater aus der Zone auswählen, den Sie gerade erreichen können. Somit bleiben Sie immer mit dieser Sprechgruppe verbunden. Einige (eher teurere) Funkgeräte unterstützen dies mit einer automatischen Roamingfunktion, bei der der jeweils stärkste Repeater an Ihrem Standort ausgewählt wird.

\begin{hinweis}
 Kanäle, die keiner Zone zugeordnet wurden, können nicht im Funkgerät ausgewählt werden. Es ist aber problemlos möglich, einen Kanal mehreren Zonen zuzuordnen.
\end{hinweis}

\subsection{Scanlisten Zusammenstellen}
Scanlisten sind einfach Listen von Kanälen, die beim Starten der Scanfunktion sequenziell beobachtet werden. Wird ein Signal auf einem der Kanäle empfangen, wird der Scan unterbrochen und es kann dann auf diesen empfangenen Ruf geantwortet werden. Diese Funktion erlaubt es mehrere Kanäle zu beobachten. Zusätzlich können Sie bei vielen Geräten ein oder zwei Prioritätskanäle definieren werden, die während des Scans häufiger \emph{besucht} und somit \emph{intensiver} beobachtet werden.

