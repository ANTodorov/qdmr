\section{Datendienste} \label{sec:data}
Da DMR von sich aus schon eine digitale Betriebsart ist, bei der meist Sprache in digitalisierter 
Form übertragen wird, ist es natürlich auch möglich reine Datendienste über DMR anzubieten. Zum 
einen gibt es einen Textnachrichtendienst, der dem SMS-Dienst der Mobiltelefone nachempfunden ist. 
Zum anderen gibt es auch die Möglichkeit, die eigene Position per DMR an das APRS\footnote{APRS 
steht für \emph{Automatic Packet Reporting System} und ermöglicht das Übertragen von kleinen 
Datensätzen über Packet-Radio wie zum Beispiel die Position, Wetter oder Textnachrichten. Mehr 
dazu erfahren sie in der \href{https://de.wikipedia.org/wiki/Automatic_Packet_Reporting_System}{Wikipedia}.} 
Netz zu übertragen.
 
\subsection{Textnachrichten (SMS)} \label{sec:textmsg}
Mit diesem Dienst können sie kurze Textnachrichten\footnote{Bis zu 144 Zeichen.} direkt an andere Teilnehmer verschicken\footnote{Sie können auch Textnachrichten an ganze Sprechgruppen versenden. Dies ist aber eher unüblich und nicht wünschenswert.}. Im Prinzip funktioniert eine Textnachricht wie ein Direktruf. Ist der andere Teilnehmer erreichbar, wird die Textnachricht übermittelt. 

Es gibt aber auch \emph{Servicenummern} (gebührenfrei). Wenn sie nun eine Nachricht an eine solche Nummer senden, können Sie bestimmte Informationen abrufen oder versenden. In Deutschland wären das:
\begin{enumerate}
 \item 262993 -- GPS und Wetter
 \begin{itemize}
  \item Wenn Sie \texttt{help} senden, erhalten daraufhin eine Auflistung aller Kommandos.
  \item Wenn Sie \texttt{wx} senden, erhalten Sie das aktuelle Wetter am Standort des Repeaters, den Sie verwenden.
  \item Wenn Sie \texttt{wx STADTNAME} senden, erhalten Sie das aktuelle Wetter für die angegebene Stadt.
  \item Wenn Sie \texttt{gps} senden, erhalten Sie die letzte Positionsinformation, die Sie zuletzt an das DMR Netz gesendet hatten.
  \item Mit \texttt{gps CALL} können Sie auch die letzte Position des angegebenen Teilnehmers abfragen.
  \item Mit \texttt{rssi} erhalten Sie vom Repeater einen Signalrapport.
 \end{itemize} 
 \item 262994 -- Repeater Informationen \& Pagernachrichten
  \begin{itemize}
   \item Wenn Sie \texttt{rpt} senden, erhalten Sie eine Liste der statisch und dynamisch abonnierten Sprechgruppen des Repeaters.  
   \item Wenn Sie \texttt{CALL NACHRICHT}, wird die angegebene Nachricht an das angegebene Call per Pager (DAPNET) geschickt.
  \end{itemize}
\end{enumerate}
 
\subsection{Positionsübermittlung (APRS via DMR)} \label{sec:aprs}
Wie im vorherigen Abschnitt schon erwähnt, ist es möglich seine Position ins DMR Netz zu senden. Diese wird dann üblicherweise direkt an das APRS-Netz weitergereicht und Ihre Position kann dann unter anderem bei \url{https://aprs.fi} abgefragt werden. Dazu ist jedoch ein DMR Funkgerät mit GPS Empfänger nötig. Aber auch diese Geräte sind in der Zwischenzeit nicht mehr teuer. Einfache DMR Handfunkgeräte mit GPS sind ab circa \EUR{120} zu haben. 

Neben dem SMS Service ist auch die Positionsübermitellung per DMR möglich. Dazu muss das GPS fähige Funkgerät so konfiguriert werden, dass die Positionsdaten auf den geeigneten Kanälen an die Nummer 262999 gesendet werden. Wie dies einzustellen ist, hängt sehr vom Hersteller des Funkgerätes ab. 