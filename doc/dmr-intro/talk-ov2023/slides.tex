\documentclass[aspectratio=169]{beamer}
\usetheme{Boadilla}

\usepackage{hyperref}
\usepackage{graphicx}
\usepackage{subcaption}
\usepackage{standalone}
\usepackage[ngerman]{babel}
\usepackage{tikz}
\usepackage{pgfplots}
\usetikzlibrary{shapes.geometric}
\usepackage{listings}
\usepackage[utf8]{inputenc}
\usepackage{tikzsymbols}
\usepackage{standalone}

\title[DMR]{Kurze Einführung in Digital Mobile Radio (DMR)}
\subtitle{Ein Mobilfunknetz für den Amateurfunk}

\author{Hannes, DM3MAT}
\institute{\texttt{dm3mat [at] darc [dot] de}}
\date{5. April 2023}

\lstset{ %
basicstyle=\tiny    % the size of the fonts that are used for the line-numbers
}


\newcommand{\repeater}[3]{%
 \node ({#1}) at ({#2}) {%
  \begin{tikzpicture}%
   \draw [black,thick] (-.25,0) -- (0,0.5) -- (0.25,0) -- (-0.25,0);%
   \draw [black,thick,domain=-45:225] plot ({0.2*cos(\x)}, {0.5+0.2*sin(\x)});%
   \draw [black,thick,domain=-45:225] plot ({0.4*cos(\x)}, {0.5+0.4*sin(\x)});%
   \node (xxx) at (0,-.2) {{#3}};%
  \end{tikzpicture}%
 } %
}

\newcommand{\activerepeater}[3]{%
 \node ({#1}) at ({#2}) {%
  \begin{tikzpicture}%
   \draw [black,thick] (-.25,0) -- (0,0.5) -- (0.25,0) -- (-0.25,0);%
   \draw [red,thick,domain=-45:225] plot ({0.2*cos(\x)}, {0.5+0.2*sin(\x)});%
   \draw [red,thick,domain=-45:225] plot ({0.4*cos(\x)}, {0.5+0.4*sin(\x)});%
   \node (xxx) at (0,-.2) {{#3}};%
  \end{tikzpicture}%
 } %
}


\newcommand{\user}[3]{%
 \node ({#1}) at ({#2}) {%
  \begin{tikzpicture}%
   \draw [black,fill=black] (-.25,0) -- (0,0.5) -- (0.25,0) -- (-0.25,0);%
   \draw [black,fill=black] (0,.5) circle (.2); %
   \node (xxx) [text width=0.6cm, align=center] at (-.35cm,-.4) {{#3}};%
  \end{tikzpicture}%
 } %
}

\newcommand{\activeuser}[3]{%
 \node ({#1}) at ({#2}) {%
  \begin{tikzpicture}%
   \draw [red,fill=red] (-.25,0) -- (0,0.5) -- (0.25,0) -- (-0.25,0);%
   \draw [red,fill=red] (0,.5) circle (.2); %
   \node (xxx) [text width=0.6cm, align=center] at (-.35cm,-.4) {{#3}};%
  \end{tikzpicture}%
 } %
}


\begin{document}
\begin{frame}
 \titlepage
\end{frame}

\begin{frame} \frametitle{Übersicht}
 \tableofcontents
\end{frame}


\section{Motivation}
\begin{frame}{Motivation}
Schnatterfunk:
\begin{itemize}
 \item Ziel: Ich möchte mit bestimmten Leuten oder auch mit Irgendjemanden reden.
 \pause\item Auf UKW: Reichweite begrenzt. Gerade mit Handfunke.
 \pause\item Lösung: Repeater.
 \pause\item Nächstes Problem: Reichweite immer noch begrenzt.
 \pause\item Lösung: Repeater vernetzen! (Echolink)
 \pause\item Immer noch Probleme: 
 \begin{itemize}
  \pause\item Wo sitzen die Leute, mit denen ich reden Will?
  \pause\item Welche Repeater sind dort in der Nähe?
  \pause\item Welche EL-Nummer haben die? 
 \end{itemize}
 \pause\item Eigentliches Problem: Was interessieren mich Repeater? Mich interessieren die Leute!
\end{itemize}
\end{frame}

\begin{frame}{Repeatertransparenz}
Eigentlich sollten FM-Relais 4 verschiedene Anwendungsfälle abdecken:
\begin{enumerate}
 \pause\item Direktes QSO mit einer bestimmten Person, egal wo diese sitzt. \pause\Sadey
 \pause\item Teilnahme an themenspezifischer Runde, egal wo die Teilnehmer sitzen. \pause\Neutrey
 \pause\item Teilnahme an regionaler Runde. \pause\Neutrey
 \pause\item QSO zum nächsten Dorf. \pause\Smiley
\end{enumerate}
\end{frame}

\begin{frame}{Repeatertransparenz}
Es wäre also schön, wenn der einzelne Repeater nicht mehr so im Zentrum stehen würde. 

\pause Wir vernetzen also die Repeater und packen wir was anderes in die Mitte:
\begin{block}{Sprechgruppe/Talkgroup}
 Eine Sprechgruppe/Talkgroup ist ein virtueller Raum/Repeater. Er existiert nicht physisch durch einen Zusammenschluss bestimmter Repeater, sondern im Netz aller Repeater. 
 
 Habe ich eine Sprechgruppe (TG) abonniert, höre ich alles, was in dieser TG gesagt wird. Sende ich dort hin, hören alle Teilnehmer meine Aussendung, egal über welchen Repeater. Sprechgruppen sind also repeatertransparent.
\end{block}
\end{frame}


\section{Ursprung}
\begin{frame}{Ursprung}
Digital Mobile Radio (DMR) hat seinen Ursprung als digitalisierter Bündelfunk/Betriebsfunk. Daher sind einige Techniken und Begriffe an diesen angelehnt. Einige dieser Techniken werden im AFu nicht verwendet (Alarm, Verschlüsselung) oder zweckentfremdet (all call).\\[0.2cm]

\pause Beispiel Flughafen (Gebäude): Es gibt eine Vielzahl an Gruppen:
\begin{itemize}
 \item Die Reinigungskolonne,
 \item die Sicherheitsleute wie Gepäckkontrolle oder Wachschutz,
 \item die Techniker,
 \item die Betriebsfeuerwehr und
 \item die Zentrale.
\end{itemize}
\pause Gleichzeitig ist so ein Flughafen ein riesiges Gelände. Das heißt, nicht alle Mitarbeiter können alle anderen Mitarbeiter direkt erreichen. Es müssen also Repeater aufgestellt werden, damit das gesamte Gelände und alle Innenräume per Funk abgedeckt sind. Daher wird häufig in jedem Gebäude mindestens ein Repeater aufgestellt. 
\end{frame}

\begin{frame}{Beispiel: Flughafen}
\centering
\includegraphics[width=\linewidth]{../fig/trunk_net_ex1.tex}
\end{frame}

\begin{frame}
\centering
\includegraphics[width=\linewidth]{../fig/trunk_net_ex2.tex}
\end{frame}

\begin{frame}
\centering
\includegraphics[width=\linewidth]{../fig/trunk_net_ex3.tex}
\end{frame}

\begin{frame}
\centering
\includegraphics[width=\linewidth]{../fig/trunk_net_ex4a.tex}
\end{frame}

\begin{frame}
\centering
\includegraphics[width=\linewidth]{../fig/trunk_net_ex4b.tex}
\end{frame}

\section{Sprechgruppen}
\begin{frame}{Nachmittagsschnatterrunde}
\begin{block}{Anwendungsbeispiel im Amateurfunk: die Nachmittagsrunde}
\begin{itemize}
 \item Eine Nachmittagsrunde findet in der regionalen Sprechgruppe statt. Z.B., in der TG 2621 \emph{Berlin/Brandenburg} kurz BB.
 \item Alle Teilnehmer befinden sich in der Region BB, außer DL3XYZ, der ist im Urlaub.
\end{itemize} 
\end{block}
\end{frame}

\begin{frame}
\centering
\includegraphics[width=\linewidth]{../fig/talkgroup_ex1a.tex}
\end{frame}

\begin{frame}
\centering
\includegraphics[width=\linewidth]{../fig/talkgroup_ex1b.tex}
\end{frame}

\begin{frame}
\centering
\includegraphics[width=\linewidth]{../fig/talkgroup_ex1c.tex}
\end{frame}


\section{Technischer Hintergrund}
\begin{frame}{Technischer Hintergrund}
DMR ist ein digitaler Funkstandard. D.h., die Sprache wird abgetastet, verlustbehaftet komprimiert (Codec) und digital übertragen (4-MFSK).\\[0.5cm]

Da digitaler Mode, können Metadaten (Quelle, Ziel, weitere Daten) parallel mit der Sprache übertragen werden.\\[0.5cm]

Moderne Codecs erlauben Sprachübertragung mit geringer Bandbreite (Datenübertragungsrate). DMR verwendet AMBE+2 (ebenso wie System Fusion).\\[0.5cm]
\end{frame}

\begin{frame}{Technischer Hintergrund}
DMR verwendet lediglich 2.45kbit/s (3.6kbit/s mit Fehlerkorrektur)\footnote{Quelle: Vortrag DJ3EI, \url{https://www.delta25.de/dmr/dmr-einf\%C3\%BChrung.pdf}}, bei 12.5kHz Kanalbreite, ist reichlich Platz.\\[0.5cm]

Wir könnten Kanalbreite halbieren (6.25kHz) oder aber zwei QSOs gleichzeitig auf einem Kanal per TDMA (time-division multiple access) ermöglichen.\\[0.5cm]
\end{frame}

\begin{frame}{TDMA - 2 QSOs auf einer Frequenz}
\begin{center}
 \includegraphics[width=0.7\linewidth]{../fig/timeslot.tex}
\end{center}
 Mit AMBE+2, können 60ms Audio in weniger als 30ms übertragen werden. D.h., wir können die Zeit in zwei Zeitschlitze aufteilen. Jeder 30ms lang.\\[0.5cm]
 
 Was/Wann Zeitschlitz 1/2 ist, bestimmt der Repeater. Er gibt den Takt vor. \\[0.5cm]
 
 Somit sind auf einem 12.5kHz Kanal zwei parallele unabhängige QSOs möglich.
\end{frame}

\begin{frame}{Qualität und Reichweite}
 Technisches Limit 150km, für Relaisbetrieb ausreichend.\\[0.5cm]
 
 Hörbare Kompressionsartefakte durch AMBE+2. Auch bei idealer Verbindung. Dennoch sehr gute Verständlichkeit, vglb. 12.5kHz FM. \\[0.5cm]
 
 Qualität beleibt auch bei schlechtem SNR erhalten. Dann aber schnelle Degradierung.
\end{frame}

\section{Konfiguration}
\begin{frame}{Konfiguration (Brandmeister)}
Die Konfiguration (sog. Codeplug erstellen) von DMR Funkgeräten ist recht komplex. \\[0.5cm]

\pause In FM ist ein Kanal ein Relais. In DMR sind Relais nicht wichtig, sondern Sprechgruppen. Wir programmieren daher logische Kanäle:
\begin{center}
\begin{tabular}{|c|c|} \hline
FM & DMR \\ \hline \hline
RX Frequenz & RX Frequenz \\
TX Ablage & TX Ablage \\
 --- & Zeitschlitz \\
(CTCSS/DCS) & Color Code \\
--- & TX Kontakt \\
--- & Empfangsgruppen \\\hline
\end{tabular}
\end{center}
\end{frame}

\begin{frame}{Konfiguration (Brandmeister)}
\begin{description}
 \item[RX/TX Frequenz] Klar.
 \item[Zeitschlitz] Hatten wir schon. Konvention: 
 \begin{itemize}
  \item TS1 überregionaler Funkverkehr
  \item TS2 regionaler/lokaler Funkverkehr
 \end{itemize}
 \item[Color Code] Meist fix 1. Vergleichbar mit CTCSS/DSC Tönen in FM. Vermeidet Probleme bei Überreichweiten.
 \item[TX Kontakt] Standard Sprechgruppe für den Kanal. Wird angerufen, wenn PTT gedrückt wird. Man kann aber auf einen empfangenden Ruf direkt antworten. 
 \item[Empfangsgruppen] Sprechgruppen, die auf einem Kanal empfangen werden sollen.
\end{description}
\end{frame}

\begin{frame}{Konfiguration -- DMR-ID}
 \begin{itemize}
  \item Jeder Teilnehmer, Sprechgruppe und Dienst benötigt eine eindeutige DMR ID. 
  \item Eine ID pro Teilnehmer, nicht Gerät! Zu bekommen bei \url{https://radioid.net}.
  \item Funktioniert wie eine Telefonnummer, incl. Prefix. Z.B., 262/263 ... Deutschland.
  \item Datenbank ID$\leftrightarrow$Call frei verfügbar. Kann auch aufs Gerät gespielt werden um ID aufzulösen.
 \end{itemize}
\end{frame}


\begin{frame}{Konfiguration -- Sprechgruppen}
\begin{description}
 \item[TG8] Regionaler Verbund von Repeatern. Nicht festgelegt.
 \item[TG9] Nur dieser Repeater. Verhält sich wie FM Repeater.
 \item[TG91] Welt weit, ist viel los dort.
 \item[TG92] Europa
 \item[TG2621] Berlin/Brandenburg
 \item[TG2629] Sachsen/Thüringen
\end{description}
\end{frame}


\begin{frame}{Konfiguration -- Kanäle}
In DMR werden pro Repeater mind. zwei Kanäle programmiert. Einen für jeden Zeitschlitz. Meist aber eher mehr. \\[0.5cm]

Ziel: Kontakt zu Sprechgruppen. 
\begin{enumerate}
 \item Liste der Sprechgruppen zusammenstellen, die uns interessieren.
 \begin{itemize}
  \item TG9, TG2621, TG2629, TG26223 (Chaosrunde CCC), ...
 \end{itemize}
 \item Liste der Repeater raus suchen.
 \begin{itemize}
  \item DB0LDS, DB0PDM, DM0LEI, ...
 \end{itemize}
\end{enumerate}
\end{frame}

\begin{frame}{Konfiguration -- Empfangsgruppen}
Überbleibsel des kommerziellen Ursprungs. Wir gruppieren alle Sprechgruppen, die uns gleichzeitig interessieren. Es macht wenig Sinn, Sprechgruppen zusammenzufassen, die nie gleichzeitig empfangen werden können. Am besten, per Region:\\[0.5cm]

\begin{description}
 \item[B/B] TG8, TG9, TG2621
 \item[Sa/Th] TG8, TG9, TG2629
 \item[CCC] TG26223
\end{description}
\end{frame}

\begin{frame}{Konfiguration -- Kanäle}
 Jetzt sind wir so weit, die Kanäle anlegen zu können:
 \begin{center}
 \begin{tabular}{|l|llllll|}\hline
 Name & RX Freq & Ablage & CC & TS & Kontakt & RX Gr. \\ \hline \hline
 DB0LDS BB  & 439.5625 & -7.6 & 1 & 2 & Berlin/Brandenburg & B/B  \\
 DB0LDS ST  & 439.5625 & -7.6 & 1 & 1 & Sachsen/Thüringen  & Sa/Th  \\
 DB0LDS CCC & 439.5625 & -7.6 & 1 & 1 & Chaosrunde         & CCC  \\
 DB0PDM BB  & 438.4000 & -7.6 & 1 & 2 & Berlin/Brandenburg & B/B  \\
 DM0LEI BB  & 439.1500 & -7.6 & 1 & 1 & Berlin/Brandenburg & B/B  \\
 DM0LEI ST  & 439.1500 & -7.6 & 1 & 2 & Sachsen/Thüringen  & Sa/Th \\ \hline
 \end{tabular}
 \end{center}
\end{frame}

\begin{frame}{Konfiguration -- Zonen}
 Da so sehr viele Kanäle entstehen, werden diese in Zonen zusammengefasst (meist nach Regionen):\\[0.5cm]
 \begin{description}
  \item[Berlin/Brandenburg] DB0LDS BB, DB0PDM BB, DB0LDS ST, DB0LDS CCC
  \item[Sachsen/Thüringen] DM0LEI BB, DM0LEI ST
 \end{description}\vspace{0.5cm}
 
 \pause Einige Geräte unterstützen auch sog. Roaming. Dabei wird einem Kanal eine Liste von alternativen Frequenzen, CCs, und TSs zugewiesen. Diese ermöglichen es, mit einer Sprechgruppe in Verbindung zu bleiben, wenn der aktuell eingestellte Kanal den Kontakt zum Repeater verliert.
\end{frame}

\section{Demo}
\begin{frame}
\begin{block}{}
 Demo...
\end{block}
\end{frame}

\section*{QDMR}
\begin{frame}{QDMR}
Da Konfiguration recht kompliziert sein kann, wird eine Software vom Hersteller verwendet (CPS, Codeplug Programming Software).
\begin{itemize}
 \item CPS vom Hersteller nur für Windows
 \item Bedienkonzept aus den 90ern. 
 \item Miese Übersetzung aus dem Chinesischen. 
 \item Sehr auf kommerzielle Anwender ausgerichtet.
 \item Keine Anbindung an User und Repeater Datenbanken.
 \item Kein Austausch von Konfigurationen zwischen Geräten oder gar Herstellern.
\end{itemize}
\end{frame}

\begin{frame}{QDMR}
 Da ich kein Windows habe, brauchte ich eine Lösung um meine DMR Geräte zu programmieren: Also eigene Software schreiben. 
 \begin{itemize}
  \item basiert auf Tool \texttt{dmrconfig} von Serge, KK6ABQ
  \item läuft auf Linux und Mac OS X
  \item verwendet gemeinsame Konfiguration für 20 verschiedene Geräte von verschiedenen Herstellern
  \item Einbindung der User DB (DMR ID $\leftrightarrow$ Calls)
  \item Einbindung der Sprechgruppen DB von Brandmeister
  \item Einbindung von Repeater Book
  \item nur 100k Zeilen C++ Code.
 \end{itemize}
\end{frame}
\end{document}